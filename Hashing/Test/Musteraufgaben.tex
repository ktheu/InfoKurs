\documentclass[addpoints]{exam}
\usepackage[utf8x]{inputenc}
\usepackage[ngerman]{babel}
\usepackage{listings}
\usepackage{babel}
\usepackage[top=1.5cm,bottom=0.5cm,headsep=0.5cm,headheight=3cm,%
left=1.5cm,right=1.5cm]{geometry}

\usepackage[T1]{fontenc}
\usepackage{booktabs} % schöne Tabellen
\usepackage{graphicx}
\usepackage{csquotes} % Anführungszeichen
\usepackage{paralist} % kompakte Aufzählungen
\usepackage{amsmath,textcomp,tikz} %diverses
\usepackage{eso-pic} % Bilder im Hintergrund
\usepackage{mdframed} % Boxen
\usepackage{multirow}


\newmdenv[linecolor=black,backgroundcolor=gray!15,
frametitle={Punktverteilung},leftmargin=1cm,
rightmargin=1cm]{infobox}

\lstset{language=Python, tabsize=4, basicstyle=\footnotesize, showstringspaces=false, mathescape=true}
\lstset{literate=%
  {Ö}{{\"O}}1
  {Ä}{{\"A}}1
  {Ü}{{\"U}}1
  {ß}{{\ss}}1
  {ü}{{\"u}}1
  {ä}{{\"a}}1
  {ö}{{\"o}}1
}
\begin{document}
\pointpoints{Punkt}{Punkte}
\bonuspointpoints{Bonuspunkt}{Bonuspunkte}
\renewcommand{\solutiontitle}{\noindent\textbf{Lösung:}%
\enspace}

\chqword{Frage}
\chpgword{Seite}
\chpword{Punkte}
\chbpword{Bonus Punkte}
\chsword{Erreicht}
\chtword{Gesamt}
\hpword{Punkte:} % Punktetabelle
\hsword{Ergebnis:}
\hqword{Aufgabe:}
\htword{Summe:}
\cellwidth{1.5em}
%\begin{center}
%\fbox{\fbox{\parbox{5.5in}{\centering
%Informatik-Klausur}}}
%\end{center}
%
%\vspace{5mm}
%
%\makebox[\textwidth]{Name:\enspace\hrulefill}
\pagestyle{headandfoot}
\runningheadrule

\newcommand\Vtextvisiblespace[1][.3em]{%
  \mbox{\kern.06em\vrule height.3ex}%
  \vbox{\hrule width#1}%
  \hbox{\vrule height.3ex}}

\newcommand{\klaubez}{Aufgaben zu Hashing}
\firstpageheader{Informatik }{\klaubez} {\thepage /\numpages}
\runningheader{Informatik }{\klaubez} {\thepage /\numpages}
\newcommand{\pfad}{c:/Users/khthe/Dropbox/Informatik/KursV2/}
%-------------------------------------------------------------------
%\printanswers
%-------------------------------------------------------------------

\begin{questions}

\renewcommand{\arraystretch}{2.0}
\setlength{\tabcolsep}{14pt}
\question[2]
Gegeben sei die Hashfunktion $f(x) = x \% 10$.
Füge die Zahlenfolge
\texttt{12 18 22 23 42 19 38}  in die Hashtabelle ein.
Benutze dabei lineares Sondieren.

\begin{tabular}{|c|c|c|c|c|c|c|c|c|c|c|}
\hline f(x) & 0 & 1 & 2 & 3 & 4 & 5 & 6 & 7 & 8 & 9 \\
\hline x     &   &    &    &   &   &    &    &   &   & \\
\hline
\end{tabular}

\ifprintanswers
Lösung:
\begin{lstlisting}
   0   1   2   3   4   5   6   7   8   9
  38   .  12  22  23  42   .   .  18  19
\end{lstlisting}
\fi



\renewcommand{\arraystretch}{2.0}
\setlength{\tabcolsep}{14pt}
\question[3]
Gegeben sei die Hashfunktion $f(x) = x \% 10$.
Füge die Zahlenfolge
\texttt{12 18 22 23 42 19 38}  in die Hashtabelle ein.
Benutze dabei quadratisches Sondieren.

\begin{tabular}{|c|c|c|c|c|c|c|c|c|c|c|}
\hline f(x) & 0 & 1 & 2 & 3 & 4 & 5 & 6 & 7 & 8 & 9 \\
\hline x     &   &    &    &   &   &    &    &   &   & \\
\hline
\end{tabular}

\ifprintanswers
Lösung:
\begin{lstlisting}
0   1   2   3   4   5   6   7   8   9
.   .  12  22  23   .  42  38  18  19
\end{lstlisting}
\fi


\renewcommand{\arraystretch}{2.0}
\setlength{\tabcolsep}{14pt}
\question[4]
Gegeben seien die beiden Hashfunktionen $f_1(x) = x \% 10$ und $f_2(x) = (2x+3) \% 10$
Füge die Zahlenfolge
\texttt{12 18 22 23 42 19 38}  in die Hashtabelle ein.
Benutze als Sondierungsverfahren Doublehashing.

\begin{tabular}{|c|c|c|c|c|c|c|c|c|c|c|}
\hline f(x) & 0 & 1 & 2 & 3 & 4 & 5 & 6 & 7 & 8 & 9 \\
\hline x     &   &    &    &   &   &    &    &   &   & \\
\hline
\end{tabular}

\ifprintanswers
Lösung:
\begin{lstlisting}
0   1   2   3   4   5   6   7   8   9
19   .  12  23   .   .  42  38  18  22
\end{lstlisting}
\fi


\renewcommand{\arraystretch}{2.0}
\setlength{\tabcolsep}{14pt}
\question[5]
Gegeben sei die Hashfunktion $f(x) = x \% 10$.
Betrachte die Hashtabelle mit den eingetragenen Werten:

\begin{tabular}{|c|c|c|c|c|c|c|c|c|c|c|}
\hline f(x) & 0 & 1 & 2 & 3 & 4 & 5 & 6 & 7 & 8 & 9 \\
\hline x     & 36 & 20 & 62 & 22 &   &    & 42 & 37 & 12 & \\
\hline
\end{tabular}

a. Durch welche Art von Hashing wurde die Hashtabelle erzeugt?
\begin{solutionbox}{1cm}
geschlossenes Hashing mit quadratischem Sondieren
\end{solutionbox}

b. Schreibe eine mögliche Reihenfolge des Einfügens auf.
\begin{solutionbox}{3cm}
\begin{lstlisting}
62 22 42 37 36 20 12
62 22 37 42 36 20 12
62 37 22 42 36 20 12
37 62 22 42 36 20 12
\end{lstlisting}
\end{solutionbox}


% -------------------------------------------------
\end{questions}
\begin{center}
%\pointtable[h][questions]
\end{center}

\end{document}
