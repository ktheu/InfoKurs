
\renewcommand{\arraystretch}{2.0}
\setlength{\tabcolsep}{14pt}
\question[4]
Gegeben seien die beiden Hashfunktionen $f_1(x) = x \% 10$ und $f_2(x) = (2x+3) \% 10$
Füge die Zahlenfolge
\texttt{12 18 22 23 42 19 38}  in die Hashtabelle ein.
Benutze als Sondierungsverfahren Doublehashing.

\begin{tabular}{|c|c|c|c|c|c|c|c|c|c|c|}
\hline f(x) & 0 & 1 & 2 & 3 & 4 & 5 & 6 & 7 & 8 & 9 \\
\hline x     &   &    &    &   &   &    &    &   &   & \\
\hline
\end{tabular}

\ifprintanswers
Lösung:
\begin{lstlisting}
0   1   2   3   4   5   6   7   8   9
19   .  12  23   .   .  42  38  18  22
\end{lstlisting}
\fi
