
\renewcommand{\arraystretch}{2.0}
\setlength{\tabcolsep}{14pt}
\question[5]
Gegeben sei die Hashfunktion $f(x) = x \% 10$.
Betrachte die Hashtabelle mit den eingetragenen Werten:

\begin{tabular}{|c|c|c|c|c|c|c|c|c|c|c|}
\hline f(x) & 0 & 1 & 2 & 3 & 4 & 5 & 6 & 7 & 8 & 9 \\
\hline x     & 36 & 20 & 62 & 22 &   &    & 42 & 37 & 12 & \\
\hline
\end{tabular}

a. Durch welche Art von Hashing wurde die Hashtabelle erzeugt?
\begin{solutionbox}{1cm}
geschlossenes Hashing mit quadratischem Sondieren
\end{solutionbox}

b. Schreibe eine mögliche Reihenfolge des Einfügens auf.
\begin{solutionbox}{3cm}
\begin{lstlisting}
62 22 42 37 36 20 12
62 22 37 42 36 20 12
62 37 22 42 36 20 12
37 62 22 42 36 20 12
\end{lstlisting}
\end{solutionbox}
