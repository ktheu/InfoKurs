\question[2]
Wie werden die Zustände (4, 10)  und  (5, 12) vom A*-Algorithmus
bewertet, wenn für die Fortwärtskosten die euklidsche Distanz zum Ziel verwendet
wird? (S = Start, E = Ziel, x = Wand, die Punkte sollen das Abzählen erleichtern.)
\begin{lstlisting}
xxxxxxxxxxxxxxxxxx
x...........E....x
x................x
x..xxxx..xxxxx...x
x..........S.x...x
x................x
xxxxxxxxxxxxxxxxxx
\end{lstlisting}
\begin{solutionbox}{2cm}
\begin{lstlisting}
(4, 10) 4.6
(5, 12) 6.0
\end{lstlisting}
\end{solutionbox}
