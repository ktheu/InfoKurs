\question[2]
Wie werden die Zustände (3,8) und (6,9) vom A*-Algorithmus
bewertet, wenn für die Fortwärtskosten die euklidsche Distanz zum Ziel verwendet
wird? Runde die Ergebnisse auf 2 Dezimalstellen.
(S = Start, E = Ziel, x = Wand, die Punkte sollen das Abzählen erleichtern.)

Hinweis zu den Koordinaten: (0,0) ist das obere linke Eck. Die erste Zahl
ist die Zeile, die zweite die Spalte, d.h. im Beispiel ist das E an der Koordinate (1,12).
\begin{lstlisting}
xxxxxxxxxxxxxxxxxx
x...........E....x
x................x
x..xxxxx.xxxxx...x
x.....Sx.....x...x
x......xx........x
x................x
xxxxxxxxxxxxxxxxxx
\end{lstlisting}
\begin{solutionbox}{2cm}
\begin{lstlisting}
(3, 8) 13.47
(6, 9) 10.83
\end{lstlisting}
\end{solutionbox}
