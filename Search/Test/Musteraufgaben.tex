\documentclass[addpoints]{exam}
\usepackage[utf8x]{inputenc}
\usepackage[ngerman]{babel}
\usepackage{listings}
\usepackage{babel}
\usepackage[top=1.5cm,bottom=0.5cm,headsep=0.5cm,headheight=3cm,%
left=1.5cm,right=1.5cm]{geometry}

\usepackage[T1]{fontenc}
\usepackage{booktabs} % schöne Tabellen
\usepackage{graphicx}
\usepackage{csquotes} % Anführungszeichen
\usepackage{paralist} % kompakte Aufzählungen
\usepackage{amsmath,textcomp,tikz} %diverses
\usepackage{eso-pic} % Bilder im Hintergrund
\usepackage{mdframed} % Boxen
\usepackage{multirow}


\newmdenv[linecolor=black,backgroundcolor=gray!15,
frametitle={Punktverteilung},leftmargin=1cm,
rightmargin=1cm]{infobox}

\lstset{language=Python, tabsize=4, basicstyle=\footnotesize, showstringspaces=false, mathescape=true}
\lstset{literate=%
  {Ö}{{\"O}}1
  {Ä}{{\"A}}1
  {Ü}{{\"U}}1
  {ß}{{\ss}}1
  {ü}{{\"u}}1
  {ä}{{\"a}}1
  {ö}{{\"o}}1
}
\begin{document}
\pointpoints{Punkt}{Punkte}
\bonuspointpoints{Bonuspunkt}{Bonuspunkte}
\renewcommand{\solutiontitle}{\noindent\textbf{Lösung:}%
\enspace}

\chqword{Frage}
\chpgword{Seite}
\chpword{Punkte}
\chbpword{Bonus Punkte}
\chsword{Erreicht}
\chtword{Gesamt}
\hpword{Punkte:} % Punktetabelle
\hsword{Ergebnis:}
\hqword{Aufgabe:}
\htword{Summe:}
\cellwidth{1.5em}
%\begin{center}
%\fbox{\fbox{\parbox{5.5in}{\centering
%Informatik-Klausur}}}
%\end{center}
%
%\vspace{5mm}
%
%\makebox[\textwidth]{Name:\enspace\hrulefill}
\pagestyle{headandfoot}
\runningheadrule

\newcommand\Vtextvisiblespace[1][.3em]{%
  \mbox{\kern.06em\vrule height.3ex}%
  \vbox{\hrule width#1}%
  \hbox{\vrule height.3ex}}

\newcommand{\klaubez}{Aufgaben zu Search}
\firstpageheader{Informatik }{\klaubez} {\thepage /\numpages}
\runningheader{Informatik }{\klaubez} {\thepage /\numpages}
\newcommand{\pfad}{c:/Users/khthe/Dropbox/Informatik/KursV2}
%-------------------------------------------------------------------
%\printanswers
%-------------------------------------------------------------------

\begin{questions}

\question[1]
% maze_01.txt
Welche Weglänge errechnet der im Unterricht vorgestellte bfs-Algorithmus
vom Start S zum Ziel E? (Die Punkte sind zum besseren Abzählen eingetragen).
\begin{lstlisting}
xxxxxxxxxxxxxxxxxx
x............E...x
x................x
x..xxxx..xxxxx...x
x................x
x..S.............x
xxxxxxxxxxxxxxxxxx
\end{lstlisting}
\begin{solutionbox}{4cm}
\begin{lstlisting}
xxxxxxxxxxxxxxxxxx
x......ooooooE   x
x......o.......  x
x..xxxxo.xxxxx.. x
x..ooooo.........x
x..S.............x
xxxxxxxxxxxxxxxxxx      Weglänge = 14
\end{lstlisting}
\end{solutionbox}

\question[2]
% maze_01.txt
Welche Weglänge errechnet der im Unterricht vorgestellte dfs-Algorithmus
vom Start S zum Ziel E? (Die Punkte sind zum besseren Abzählen eingetragen).
\begin{lstlisting}
xxxxxxxxxxxxxxxxxx
x............E...x
x................x
x..xxxx..xxxxx...x
x................x
x..S.............x
xxxxxxxxxxxxxxxxxx
\end{lstlisting}
\begin{solutionbox}{4cm}
\begin{lstlisting}
xxxxxxxxxxxxxxxxxx
x.ooo.ooo.   Eo. x
x.o.ooo.o.   .o. x
x.oxxxx.oxxxxxo. x
x.oo....o.ooo.o. x
x .S....ooo.ooo. x
xxxxxxxxxxxxxxxxxx      Weglänge = 30
\end{lstlisting}
\end{solutionbox}

\question[2]
% maze_01.txt
Das 'x' in der linken oberen Ecke entspricht dem Zustand (0/0). Der im
im Unterricht vorgestellte Greedy-Algorithmus untersucht im Laufe seiner
Arbeit den Zustand (5,8). Dabei werden zwei neue Folgezustände
mit der euklidschen Heuristik bewertet. Wie lauten die Folgezustände und
ihre Bewertungen?
\begin{lstlisting}
xxxxxxxxxxxxxxxxxx
x............E...x
x................x
x..xxxx..xxxxx...x
x................x
x..S.............x
xxxxxxxxxxxxxxxxxx
\end{lstlisting}
\begin{solutionbox}{4cm}
\begin{lstlisting}
(4, 8) 5.83
(5, 9) 5.66
\end{lstlisting}
\end{solutionbox}

\question[2]
Wie werden die Zustände (2, 6)  und  (4, 7) vom A*-Algorithmus
bewertet, wenn für die Fortwärtskosten die Manhattendistanz zum Ziel verwendet
wird? (S = Start, E = Ziel, x = Wand, die Punkte sollen das Abzählen erleichtern)
\begin{lstlisting}
xxxxxxxxxxxxxxxxxx
x............E...x
x................x
x..xxxx..xxxxx...x
x................x
x..S.............x
xxxxxxxxxxxxxxxxxx
\end{lstlisting}
\begin{solutionbox}{4cm}
\begin{lstlisting}
(2, 6)  16
(4, 7)  14
\end{lstlisting}
\end{solutionbox}

\question[1]
Zeichne ein Szenario, bei dem dfs einen Weg der Länge 7 findet.

\begin{solutionbox}{5cm}
\begin{lstlisting}
xxxxx
x E x
x o x
x o x
x o x
x o x
x o x
x o x
x S x
xxxxx
weglänge = 7
\end{lstlisting}
\end{solutionbox}

\question[2]
% maze_04.txt
Zeichne in das Szenario Wände so ein, dass greedy (mindestens) doppelt so
lange benötigt wie der optimale Weg.
\begin{lstlisting}
xxxxxxxxxxx
x.........x
x.........x
x.........x
x.........x
x........Ex
x.S.......x
x.........x
xxxxxxxxxxx
\end{lstlisting}
\begin{solutionbox}{4cm}
\begin{lstlisting}
xxxxxxxxxxx
x         x
x    xxx  x
x      x  x
x      x  x
x      x Ex
x Sxxxxx  x
x         x
xxxxxxxxxxx  greedy = 20, optimal = 10
\end{lstlisting}
\end{solutionbox}






% -------------------------------------------------
\end{questions}
\begin{center}
%\pointtable[h][questions]
\end{center}

\end{document}
