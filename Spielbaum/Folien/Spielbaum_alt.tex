%\documentclass[12pt,handout]{beamer}
%\documentclass{beamer}
\usepackage[ngerman]{babel}
\usepackage[utf8]{inputenc}
\usepackage{amsmath}
\usepackage{amssymb}
\usepackage{listings} 
\usepackage{stmaryrd}
\lstset{language=Python, tabsize=4, showstringspaces=false,basicstyle=\footnotesize,mathescape=true}
\lstset{literate=%
  {Ö}{{\"O}}1
  {Ä}{{\"A}}1
  {Ü}{{\"U}}1
  {ß}{{\ss}}1
  {ü}{{\"u}}1
  {ä}{{\"a}}1
  {ö}{{\"o}}1
}  
\usepackage{mathtools}
\usepackage{ulem}
\usepackage{tikz}

\usetheme{Boadilla}
\mode<presentation>{
\useoutertheme[subsection=false]{miniframes}
\useinnertheme{rectangles}
%\usecolortheme{crane}
}
\parskip 10pt



\begin{document}
\title{Informatik}   
\author{Spielbaum} 
\date{}
\frame{\titlepage} 
\small
%---
\begin{frame}[fragile]

In einem \textbf{Mehrwegebaum} kann jeder Knoten eine variable Anzahl von Söhnen besitzen. 


\includegraphics[width=10cm]{bild1a.png} 

\end{frame}


\begin{frame}[fragile]
Ein \textbf{Spielbaum} ist ein Mehrwegebaum mit zwei Typen von Knoten: Minimum-Knoten und
Maximum-Knoten.  \pause

Die Knoten repräsentieren Spielstellungen.  \pause

Die Kanten zwischen den Knoten repräsentieren Spielzüge. \pause

Der Spielbaum eignet sich für Zwei-Personen Nullsummenspiele wie Schach, Dame, TicTacToe, wo 2 Personen
 abwechselnd ziehen und der Gewinn des einen Spielers dem Verlust des anderen Spielers entspricht. \pause
 
Der Wert eines Blattes wird bestimmt durch eine statische Stellungsbewertung. \pause
Der Wert eines Minimum-Knotens ist das Minimum der Werte seiner Söhne. 
Der Wert eines Maximum-Knotens ist das Maximum der Werte seiner Söhne. 
\end{frame}

\begin{frame}[fragile]
Beispiel für einen Spielbaum

\includegraphics[width=12cm]{bild10.png} 

\end{frame}

\begin{frame}[fragile]


\includegraphics[width=12cm]{bild11.png} 

Reihenfolge, in der die Knoten ihre 
Bewertung erhalten:

e:2 f:3 b:2 g:4 h:-2 i:2 c:-2 j:1 d:1 a:2 \\
bester zug: b

\end{frame}
\begin{frame}[fragile]
\includegraphics[height=7.5cm]{minmax.png}

\tiny{Quelle: CSMM.101x Artificial Intelligence - Columbia University}
\end{frame}

\begin{frame}[fragile]
\begin{lstlisting}[basicstyle=\tiny]
inf = float('inf')
def maximize(state):
    if terminal_test(state):
        return None,evaluation(state)
    v, bestChild = -inf, None
    for child in nextstates(state):
        _,utility = minimize(child)
        if utility > v:
            bestChild,v = child,utility
    return bestChild,v

def minimize(state):
    if terminal_test(state):
        return None,evaluation(state)
    v, bestChild = inf, None
    for child in nextstates(state):
        _,utility = maximize(child)
        if utility < v:
            bestChild,v = child,utility
    return bestChild,v

# bester folgezug für maximizer nach state
bestnext, _ = maximize(state)

# bester folgezug für minimizer nach state
bestnext, _ = minimize(state)
\end{lstlisting} 
\end{frame}

\begin{frame}[fragile]
\begin{minipage}[c]{7cm}
\begin{lstlisting}[basicstyle=\tiny]
inf = float('inf')
def maximize(state):
    print(state)
    if terminal_test(state):
        return None,evaluation(state)
    v, bestChild = -inf, None
    for child in nextstates(state):
        _,utility = minimize(child)
        if utility > v:
            bestChild,v = child,utility
    print(state,bestChild,v)
    return bestChild,v

def minimize(state):
    print(state)
    if terminal_test(state):
        return None,evaluation(state)
    v, bestChild = inf, None
    for child in nextstates(state):
        _,utility = maximize(child)
        if utility < v:
            bestChild,v = child,utility
    print(state,bestChild,v)
    return bestChild,v

nxt = {'a':list('bc'),'b':list('de'),'c':list('fg')}
blatt = {'d':10,'e':8,'f':4,'g':50}
maximize('a')
\end{lstlisting} 
\end{minipage} 
\begin{minipage}[c]{4.5cm}
\begin{lstlisting}[basicstyle=\tiny]
def nextstates(state):
    return nxt[state]
        
def terminal_test(state):
    return state in blatt

def evaluation(state):
    return blatt[state]
\end{lstlisting} 
\small{Was erscheint auf der Konsole?} \pause
\begin{lstlisting}[basicstyle=\tiny]
a
b
d
e
b e 8
c
f
g
c f 4
a b 8
\end{lstlisting} 
\end{minipage} 
\end{frame}

\begin{frame}[fragile]
Für ein konkretes Spiel müssen wir uns entscheiden, wie wir eine Spielstellung repräsentieren. Anschließend müssen wir folgende Funktionen implementieren.
\begin{lstlisting}[basicstyle=\tiny]
def nextstates(state):
    '''
    state: Spielstellung
    returns: Liste mit möglichen Folgestellungen
    '''
    pass
    
        
def terminal_test(state):
    '''
    state: Spielstellung
    returns: True, wenn Stellung ein Blatt ist
    '''
    pass

def evaluation(state):
    '''
    state: Spielstellung
    returns: Zahl, die den Wert der Stellung wiedergibt
    '''
    pass
\end{lstlisting} 
\end{frame}

\begin{frame}[fragile]
Beispiel: Tic-Tac-Toe

Es gibt mehrere Möglichkeiten, eine Spielstellung zu repräsentieren.
\begin{lstlisting} 
x . . 
. o . 
. . . 
\end{lstlisting}  \pause
- Matrix mit Zeichen: \texttt{[['x','.','.'],['.','o','.'],['.','.','.']]} \\
- Matrix mit Zahlen:   \texttt{[[1,0,0],[0,-1,0],[0,0,0]]} \\
- Liste mit Tupeln: \texttt{[(0,0),(1,1)]} \\
- Liste mit Zahlen: \texttt{[0,4]} \\
usw.

\end{frame}

\begin{frame}[fragile]
Beispiel: Liste mit Zahlen, d.h. die Felder des Spielfeldes werden von 0 bis 8 durchnummeriert.
\begin{lstlisting} [basicstyle=\tiny]
def nextstates(state): $\pause$
    stateset = set(state)
    temp = []
    for i in range(9):
        if i not in stateset:
            nxt = state[:]
            nxt.append(i)
            temp.append(nxt)
    return temp     $\pause$

def evaluation(state): 
    '''
    state: Stellung
    returns: Zahl, die den Wert der Stellung wiedergibt
       +1 Maximizer gewinnt, -1 Minimizer gewinnt, 0 Unentschieden
    '''
    # your code here
    pass

def terminal_test(state):  $\pause$
    w = evaluation(state)
    return w $==$ 1 or w $==$ -1 or len(state) $==$ 9 $\pause$


\end{lstlisting} 




\end{frame}



\begin{frame}[fragile]
Der Alpha-Beta Algorithmus versucht Teile des Baumes abzuschneiden (pruning), die erkennbar nicht mehr durchsucht werden müssen.

\includegraphics[width=12cm]{bild12.png} 
\end{frame}

\begin{frame}[fragile]
\includegraphics[width=12cm]{bild13.png} 
\end{frame}


\begin{frame}[fragile]
\includegraphics[width=10cm]{alpha_beta.png} 

\tiny{Quelle: CSMM.101x Artificial Intelligence - Columbia University}
\end{frame}


\begin{frame}[fragile]
\begin{minipage}[t]{7.5cm}
\begin{lstlisting}[basicstyle=\tiny]
inf = float('inf')
def maximize(state,alpha,beta):
    if terminal_test(state):
        return None,evaluation(state)
    v, bestChild = -inf, None
    for child in nextstates(state):
        _,utility = minimize(child,alpha,beta)
        if utility > v:
            v, bestChild = utility, child
        if v >= beta:
            break
        if v > alpha:
            alpha = v
    return bestChild, v

def minimize(state,alpha,beta):
    if terminal_test(state):
        return None,evaluation(state)
    v, bestChild = inf, None
    for child in nextstates(state):
        _,utility = maximize(child,alpha,beta)
        if utility < v:
            v, bestChild = utility, child
        if v <= alpha:
            break
        if v < beta:
            beta = v
    return bestChild, v 

# bester folgezug für maximizer nach state
bestnext, _ = maximize(state,-inf,inf)
\end{lstlisting} \pause
\end{minipage} 
%\begin{minipage}[t]{4cm}
%alpha: die beste bisher untersuchte Option entlang des Pfads zur Wurzel für den Maximizer. 
%~~\\~~\\
%beta: die beste bisher untersuchte Option entlang des Pfads zur Wurzel  für den Minimizer.
%~~\\~~\\
%pruning: \\
%bei max, wenn $v \ge \beta$ \\
%bei min, wenn $v \le \alpha$
%\end{minipage} 
\end{frame}

\begin{frame}[fragile]

$\alpha$ und $\beta$ werden von oben nach unten durchgereicht. Bei den max-Knoten wird $\beta$ nicht verändert, bei den min-Knoten wird $\alpha$ nicht verändert. Wir notieren deshalb nur die $\alpha$-Werte bei den max-Knoten und nur die $\beta$-Werte bei den min-Knoten. An den Blätter notieren wir keine $\alpha$ oder $\beta$-Werte.

Der Algorithmus ist eine Tiefensuche, die abwärts- und aufwärts-Bewegungen macht. Abwärts erben die Kinder ihre  $\alpha$ und $\beta$-Werte von den Großeltern. Wenn aufwärts ein Eltern-Knoten bei einem Kind etwas sieht, was er haben will, nimmt er es von dem Kind.

Wenn an einer Stelle entdeckt wird, dass $\alpha$ größer-gleich dem darüber stehenden $\beta$ ist oder
dass $\beta$ kleiner-gleich dem darüber stehenden  $\alpha$ ist, wird gepruned.

Der Auftraggeber des pruning ist der Vater des Knotens, bei dem das pruning erfolgt.
\end{frame}

\begin{frame}[fragile]
\includegraphics[width=12cm]{bild14.png} 



\end{frame}

\begin{frame}[fragile]
\includegraphics[width=12cm]{bild15.png} 

Reihenfolge der statischen Bewertungen, \# für pruning:
\begin{lstlisting}
k l m # o p # s t u # 
bester Zug d
\end{lstlisting} 
\end{frame}



 \end{document}