\documentclass[addpoints]{exam}
\usepackage[utf8x]{inputenc}
\usepackage[ngerman]{babel}
\usepackage{listings}
\usepackage{babel}
\usepackage[top=1.5cm,bottom=0.5cm,headsep=0.5cm,headheight=3cm,%
left=1.5cm,right=1.5cm]{geometry}

\usepackage[T1]{fontenc}
\usepackage{booktabs} % schöne Tabellen
\usepackage{graphicx}
\usepackage{csquotes} % Anführungszeichen
\usepackage{paralist} % kompakte Aufzählungen
\usepackage{amsmath,textcomp,tikz} %diverses
\usepackage{eso-pic} % Bilder im Hintergrund
\usepackage{mdframed} % Boxen
\usepackage{multirow}


\newmdenv[linecolor=black,backgroundcolor=gray!15,
frametitle={Punktverteilung},leftmargin=1cm,
rightmargin=1cm]{infobox}

\lstset{language=Python, tabsize=4, basicstyle=\footnotesize, showstringspaces=false, mathescape=true}
\lstset{literate=%
  {Ö}{{\"O}}1
  {Ä}{{\"A}}1
  {Ü}{{\"U}}1
  {ß}{{\ss}}1
  {ü}{{\"u}}1
  {ä}{{\"a}}1
  {ö}{{\"o}}1
}
\begin{document}
\pointpoints{Punkt}{Punkte}
\bonuspointpoints{Bonuspunkt}{Bonuspunkte}
\renewcommand{\solutiontitle}{\noindent\textbf{Lösung:}%
\enspace}

\chqword{Frage}
\chpgword{Seite}
\chpword{Punkte}
\chbpword{Bonus Punkte}
\chsword{Erreicht}
\chtword{Gesamt}
\hpword{Punkte:} % Punktetabelle
\hsword{Ergebnis:}
\hqword{Aufgabe:}
\htword{Summe:}
\cellwidth{1.5em}
%\begin{center}
%\fbox{\fbox{\parbox{5.5in}{\centering
%Informatik-Klausur}}}
%\end{center}
%
%\vspace{5mm}
%
%\makebox[\textwidth]{Name:\enspace\hrulefill}
\pagestyle{headandfoot}
\runningheadrule

\newcommand\Vtextvisiblespace[1][.3em]{%
  \mbox{\kern.06em\vrule height.3ex}%
  \vbox{\hrule width#1}%
  \hbox{\vrule height.3ex}}

\newcommand{\klaubez}{Aufgaben zu Stringmethoden}
\firstpageheader{Informatik }{\klaubez} {\thepage /\numpages}
\runningheader{Informatik }{\klaubez} {\thepage /\numpages}
\newcommand{\pfad}{c:/Users/khthe/Dropbox/Informatik/KursV2/Python}
%-------------------------------------------------------------------
%\printanswers
%-------------------------------------------------------------------

\begin{questions}
\question[3]
Was erscheint auf der Konsole?
\begin{lstlisting}
a = [4,5,8]
b = a.pop()
c = a.remove(4)
a.append(9)
a.insert(1,2)
print(a,b,c)
\end{lstlisting}
\begin{solutionbox}{1cm}
\texttt{[5, 2, 9] 8 None}
\end{solutionbox}

\question[4] Was erscheint auf der Konsole? Schreibe error für einen Fehler und ein Minuszeichen
für eine Leerzeile.
\begin{lstlisting}
s = 'cDAdaBAA'
print(s.swapcase())
print(s.index('b'))
print(s.find('A'))
print(s.count('a'))
\end{lstlisting}
\begin{solutionbox}{3cm}
\begin{lstlisting}
CdaDAbaa
error
2
1
\end{lstlisting}
\end{solutionbox}

\question[2] Was erscheint auf der Konsole?
\begin{lstlisting}
s = 'acBBa'
a = s.replace('a','c').capitalize().swapcase().replace('C','1')
print(a)
\end{lstlisting}
\begin{solutionbox}{2cm}
\begin{lstlisting}
c1BB1
\end{lstlisting}
\end{solutionbox}

\question[2] Was erscheint auf der Konsole?
\begin{lstlisting}
s = 'dEbbA'
a = s.capitalize().swapcase().replace('b','c')
print(a)
\end{lstlisting}
\begin{solutionbox}{2cm}
\begin{lstlisting}
dEBBA
\end{lstlisting}
\end{solutionbox}

\question[2] Was erscheint auf der Konsole? Schreibe error für einen Fehler.
\begin{lstlisting}
a = '12ab'
print(a.isalpha() or a.isdigit())
b = '1234'
print(b.isdigit() and b.isalnum())
\end{lstlisting}
\begin{solutionbox}{2cm}
\begin{lstlisting}
False
True
\end{lstlisting}
\end{solutionbox}

\question[2] Was erscheint auf der Konsole? Schreibe error für einen Fehler.
\begin{lstlisting}
a =  125
print(a.isalpha() or a.isdigit())
b = 'abc'
print(b.isdigit() or b.isalnum())
\end{lstlisting}
\begin{solutionbox}{2cm}
\begin{lstlisting}
error
True
\end{lstlisting}
\end{solutionbox}

\question[3] Ergänge den fehlenden Format-String
\begin{lstlisting}
s1, x1 = 'Alice', 5.7876
s2, x2 = 'Bob', 14.4421
f =
print(f.format(s1,x1))
print(f.format(s2,x2))

Ausgabe: (blanks sind fürs Abzählen durch Punkte ersetzt)
Da.ist.Alice...mit.der.Zahl...5.79
Da.ist.Bob.....mit.der.Zahl..14.44
\end{lstlisting}
\begin{solutionbox}{2cm}
\begin{lstlisting}
f = 'Da ist {:7} mit der Zahl {:6.2f}'
\end{lstlisting}
\end{solutionbox}

\question[2] Ergänge den fehlenden Format-String
\begin{lstlisting}
s1, x1 = 'Alice', 5.7876
s2, x2 = 'Bob', 14.4421
f =
print(f.format(s1,x1))
print(f.format(s2,x2))

Ausgabe: (blanks sind fürs Abzählen durch Punkte ersetzt)
!......Alice.!.5.788...!
!........Bob.!.14.442..!
\end{lstlisting}
\begin{solutionbox}{2cm}
\begin{lstlisting}
f = '! {:>10} ! {:<7.3f} !'
\end{lstlisting}
\end{solutionbox}

\question[3] Implementiere die Funktion func.

\begin{lstlisting}
def func(s, teil):
    '''
    s, teil: Strings
    returns True, wenn s mit teil beginnt oder endet und die Länge
       von s nicht größer ist als die doppelte Länge von teil

    Beispiele:
    >>> func('abc','ab')
    True
    >>> func('abc','abcccc')
    False

    '''
\end{lstlisting}
\begin{solutionbox}{3cm}
\begin{lstlisting}
def func(s, teil):
    return (s.startswith(teil) or s.endswith(teil)) and not len(s) > 2 * len(teil)
\end{lstlisting}
\end{solutionbox}

\question[3] Implementiere die Funktion func.

\begin{lstlisting}
def func(s):
    '''
    s: String
    returns True, wenn s mindestens die Länge 5 hat und die ersten beiden
       Zeichen alphanumerisch und die letzten beiden Zeichen Ziffern sind.

    Beispiele:
    >>> func('R2be-09')
    True
    >>> func('A?2b09')
    False
    '''

\end{lstlisting}
\begin{solutionbox}{3cm}
\begin{lstlisting}
def func(s):
    return len(s) >= 5 and s[:2].isalnum() and s[-2:].isdigit()
\end{lstlisting}
\end{solutionbox}





% -------------------------------------------------
\end{questions}
\begin{center}
%\pointtable[h][questions]
\end{center}

\end{document}
