\documentclass[addpoints]{exam}
\usepackage[utf8x]{inputenc}
\usepackage[ngerman]{babel}
\usepackage{listings}
\usepackage{babel}
\usepackage[top=1.5cm,bottom=0.5cm,headsep=0.5cm,headheight=3cm,%
left=1.5cm,right=1.5cm]{geometry}

\usepackage[T1]{fontenc}
\usepackage{booktabs} % schöne Tabellen
\usepackage{graphicx}
\usepackage{csquotes} % Anführungszeichen
\usepackage{paralist} % kompakte Aufzählungen
\usepackage{amsmath,textcomp,tikz} %diverses
\usepackage{eso-pic} % Bilder im Hintergrund
\usepackage{mdframed} % Boxen
\usepackage{multirow}


\newmdenv[linecolor=black,backgroundcolor=gray!15,
frametitle={Punktverteilung},leftmargin=1cm,
rightmargin=1cm]{infobox}

\lstset{language=Python, tabsize=4, basicstyle=\footnotesize, showstringspaces=false, mathescape=true,commentstyle=\fontfamily{fve}  }
\lstset{literate=%
  {Ö}{{\"O}}1
  {Ä}{{\"A}}1
  {Ü}{{\"U}}1
  {ß}{{\ss}}1
  {ü}{{\"u}}1
  {ä}{{\"a}}1
  {ö}{{\"o}}1
}
\begin{document}
\pointpoints{Punkt}{Punkte}
\bonuspointpoints{Bonuspunkt}{Bonuspunkte}
\renewcommand{\solutiontitle}{\noindent\textbf{Lösung:}%
\enspace}

\chqword{Frage}
\chpgword{Seite}
\chpword{Punkte}
\chbpword{Bonus Punkte}
\chsword{Erreicht}
\chtword{Gesamt}
\hpword{Punkte:} % Punktetabelle
\hsword{Ergebnis:}
\hqword{Aufgabe:}
\htword{Summe:}
\cellwidth{1.5em}
%\begin{center}
%\fbox{\fbox{\parbox{5.5in}{\centering
%Informatik-Klausur}}}
%\end{center}
%
%\vspace{5mm}
%
%\makebox[\textwidth]{Name:\enspace\hrulefill}
\pagestyle{headandfoot}
\runningheadrule

\newcommand\Vtextvisiblespace[1][.3em]{%
  \mbox{\kern.06em\vrule height.3ex}%
  \vbox{\hrule width#1}%
  \hbox{\vrule height.3ex}}

\newcommand{\klaubez}{Aufgaben zu Lists}
\firstpageheader{Informatik }{\klaubez} {\thepage /\numpages}
\runningheader{Informatik }{\klaubez} {\thepage /\numpages}
\newcommand{\pfad}{c:/Users/khthe/Dropbox/Informatik/KursV2/Python}
%-------------------------------------------------------------------
%\printanswers
%-------------------------------------------------------------------

\begin{questions}

\question[3]
Was erscheint auf der Konsole?
\begin{lstlisting}
a = [4,12,5,20,9]
b = [max(a),min(a),len(a)]
c = [7 in a, 10 not in a]
d = a + b + c
print(d)
\end{lstlisting}
\begin{solutionbox}{1cm}
\texttt{[4, 12, 5, 20, 9, 20, 4, 5, False, True]}
\end{solutionbox}

\question[3]
Was erscheint auf der Konsole?
\begin{lstlisting}
a = [2,5,4]
b = a * max(a)
c = b[2::2]
print(c)
\end{lstlisting}
\begin{solutionbox}{1cm}
\texttt{[4, 5, 2, 4, 5, 2, 4]}
\end{solutionbox}

\question[4]
In dieser Aufgabe soll, ausgehend von einer Liste a, mittels slicing oder
indexing der angegebene Wert erreicht werden.
Beispiel: \texttt{a = ['abc',(4,7)]}.
 Der Wert \texttt{'ab'} kann mittels \texttt{a[0][0:2]} erreicht werden, der Wert
\texttt{7} kann mittels \texttt{a[1][1]} erreicht werden.

\texttt{a = [42,'abc',('de',102),['uvw']]} \\
Mit welchem Ausdruck kann man, von a ausgehend, folgende Werte erreichen:   \\
\begin{tabular}{lllll}
a. 42 & b. 102 & c. 'c' & d. 'w' \\
\end{tabular}
\begin{solutionbox}{1cm}
\texttt{a. a[0] ~~ b. a[2][1] ~~ c. a[1][2] ~~ d. a[3][0][2]}
\end{solutionbox}

\question[4]
Schreibe eine Funktion \texttt{suche}, der eine nicht leere Liste
mit ganzen Zahlen übergeben wird und die die Zahl in der Liste findet,
die zur 42 den geringsten Abstand hat. Hinweis: den Abstand zwischen
zwei Zahlen x und y kann man mit \texttt{abs(x-y)} berechnen.

\begin{solutionbox}{9cm}
\begin{lstlisting}
def suche(a):
    best = a[0]
    bestAbstand = abs(a[0]-42)
    for x in a:
        if abs(x-42) < bestAbstand:
            best = x
            bestAbstand = abs(x-42)
    return best
\end{lstlisting}
\end{solutionbox}

\question[1]
Ergänze die fehlende rechte Seite, so dass sich der String s zu
\texttt{'c3po'} auswertet.

\begin{lstlisting}
a =  ['c','3','p','o']
s = ???
\end{lstlisting}
\begin{solutionbox}{1cm}
\texttt{s = ''.join(a)}
\end{solutionbox}

\question[3]
Was erscheint auf der Konsole?
\begin{lstlisting}
a = [4,5,8]
b = a.pop()
c = a.remove(4)
a.append(9)
a.insert(1,2)
print(a,b,c)
\end{lstlisting}
\begin{solutionbox}{1cm}
\texttt{[5, 2, 9] 8 None}
\end{solutionbox}

\question[3]
Was erscheint auf der Konsole?
\begin{lstlisting}
a = [3,11,5]
a.insert(2,7)
b = a.pop()
a.append(9)
c = a.index(11)
print(a,b,c)
\end{lstlisting}

\begin{solutionbox}{1cm}
\texttt{[3, 11, 7, 9] 5 1}
\end{solutionbox}

\question[2]

Gegeben sei die Liste \texttt{d = [4,7,5,9]}. Schreibe Anweisungen für:

a. Die Liste \texttt{d} soll im absteigender Reihenfolge sortiert werden. \\
b. Die Liste \texttt{e} soll die Elemente der Liste \texttt{d} in in sortierter Reihenfolge
enthalten. Die Liste \texttt{d} soll unverändert bleiben.

\begin{solutionbox}{3cm}
\begin{lstlisting}
a. d.sort(reverse = True)
b. e = sorted(d)
\end{lstlisting}
\end{solutionbox}

\question[3]
Was erscheint auf der Konsole?
\begin{lstlisting}
a = [1,2]
b = a
c = b[:]
a.append(3)
b.append(4)
c.append(5)
print(a,b,c)
\end{lstlisting}
\begin{solutionbox}{3cm}
[1, 2, 3, 4] [1, 2, 3, 4] [1, 2, 5]
\end{solutionbox}



% -------------------------------------------------
\end{questions}
\begin{center}
%\pointtable[h][questions]
\end{center}

\end{document}
