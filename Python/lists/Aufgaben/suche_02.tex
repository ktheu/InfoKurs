\question[4]
Schreibe eine Funktion \texttt{suche}, der eine nicht leere Liste mit
ganzen Zahlen übergeben wird und die die Zahl in der Liste findet,
die zur Zahl 100 den größten Abstand hat.
Hinweis: den Abstand zwischen zwei Zahlen x und y kann man mit
\texttt{abs(x-y)} berechnen.

Beispiel: Für die Liste \texttt{a = [105, 102, 90, 120, 117]} gibt die Funktion
die Zahl \texttt{120} zurück.

\begin{solutionbox}{9cm}
\begin{lstlisting}
def suche(a):
    best = a[0]
    bestAbstand = abs(a[0]-100)
    for x in a:
        if abs(x-100) > bestAbstand:
            best = x
            bestAbstand = abs(x-100)
    return best
\end{lstlisting}
\end{solutionbox}
