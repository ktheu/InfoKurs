\question[4]
In dieser Aufgabe soll, ausgehend von einer Liste a, mittels slicing oder
 indexing der angegebene Wert erreicht werden.
Beispiel: \texttt{a = ['abc',(4,7)]}.
 Der Wert \texttt{'ab'} kann mittels \texttt{a[0][0:2]} erreicht werden, der Wert 
\texttt{7} kann mittels \texttt{a[1][1]} erreicht werden.

\texttt{a = [(6,),['abcd','ef'],['g','hij',4],[9,17]]} \\
Mit welchem Ausdruck kann man, von a ausgehend, folgende Werte erreichen:  \\
\begin{tabular}{lllll}
a. 6 & b. 'bcd' & c. 'g' & d. [9, 17] \\
\end{tabular}
\begin{solutionbox}{1cm}
\texttt{a. a[0][0] ~~  b. a[1][0][1:] ~~ c. a[2][0] ~~ d. a[3]}
\end{solutionbox}
