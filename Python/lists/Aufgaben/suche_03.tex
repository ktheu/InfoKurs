\question[4]
Schreibe eine Funktion \texttt{suche}, der eine nicht leere Liste mit mindestens
zwei ganzen Zahlen übergeben wird und die die Zahl in der Liste findet, die
zu ihrem rechten Nachbarn den geringsten Abstand hat. Die letzte
Zahl muss nicht berücksichtigt werden.
Hinweis: den Abstand zwischen zwei Zahlen x und y kann man mit \texttt{abs(x-y)} berechnen.
Beispiel: Für die Liste \texttt{a = [4, 9, 8, 12]} gibt die Funktion
die Zahl \texttt{9} zurück.

\begin{solutionbox}{9cm}
\begin{lstlisting}
def suche(a):
    best = a[0]
    bestAbstand = abs(a[0]-a[1])
    for i in range(len(a)-1):
        if abs(a[i]-a[i+1]) < bestAbstand:
            best = a[i]
            bestAbstand = abs(a[i]-a[i+1])
    return best
\end{lstlisting}
\end{solutionbox}
