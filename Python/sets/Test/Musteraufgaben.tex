\documentclass[addpoints]{exam}
\usepackage[utf8x]{inputenc}
\usepackage[ngerman]{babel}
\usepackage{listings}
\usepackage{babel}
\usepackage[top=1.5cm,bottom=0.5cm,headsep=0.5cm,headheight=3cm,%
left=1.5cm,right=1.5cm]{geometry}

\usepackage[T1]{fontenc}
\usepackage{booktabs} % schöne Tabellen
\usepackage{graphicx}
\usepackage{csquotes} % Anführungszeichen
\usepackage{paralist} % kompakte Aufzählungen
\usepackage{amsmath,textcomp,tikz} %diverses
\usepackage{eso-pic} % Bilder im Hintergrund
\usepackage{mdframed} % Boxen
\usepackage{multirow}


\newmdenv[linecolor=black,backgroundcolor=gray!15,
frametitle={Punktverteilung},leftmargin=1cm,
rightmargin=1cm]{infobox}

\lstset{language=Python, tabsize=4, basicstyle=\footnotesize, showstringspaces=false, mathescape=true,commentstyle=\fontfamily{fve}  }
\lstset{literate=%
  {Ö}{{\"O}}1
  {Ä}{{\"A}}1
  {Ü}{{\"U}}1
  {ß}{{\ss}}1
  {ü}{{\"u}}1
  {ä}{{\"a}}1
  {ö}{{\"o}}1
}
\begin{document}
\pointpoints{Punkt}{Punkte}
\bonuspointpoints{Bonuspunkt}{Bonuspunkte}
\renewcommand{\solutiontitle}{\noindent\textbf{Lösung:}%
\enspace}

\chqword{Frage}
\chpgword{Seite}
\chpword{Punkte}
\chbpword{Bonus Punkte}
\chsword{Erreicht}
\chtword{Gesamt}
\hpword{Punkte:} % Punktetabelle
\hsword{Ergebnis:}
\hqword{Aufgabe:}
\htword{Summe:}
\cellwidth{1.5em}
%\begin{center}
%\fbox{\fbox{\parbox{5.5in}{\centering
%Informatik-Klausur}}}
%\end{center}
%
%\vspace{5mm}
%
%\makebox[\textwidth]{Name:\enspace\hrulefill}
\pagestyle{headandfoot}
\runningheadrule

\newcommand\Vtextvisiblespace[1][.3em]{%
  \mbox{\kern.06em\vrule height.3ex}%
  \vbox{\hrule width#1}%
  \hbox{\vrule height.3ex}}

\newcommand{\klaubez}{Aufgaben zu Sets}
\firstpageheader{Informatik }{\klaubez} {\thepage /\numpages}
\runningheader{Informatik }{\klaubez} {\thepage /\numpages}
\newcommand{\pfad}{c:/Users/khthe/Dropbox/Informatik/KursV2/Python}
%-------------------------------------------------------------------
%\printanswers
%-------------------------------------------------------------------

\begin{questions}

\question[3] Was erscheint auf der Konsole?
 Notiere die einzelnen Ausgaben aus Platzgründen horizontal mit Komma getrennt.
\begin{lstlisting}
m1 = {4,5,7}
m2 = {4,4}
m3 = {4}
print(m2 <= m1)
print(len(m2))
print(m3 < m2)
\end{lstlisting}
\begin{solutionbox}{2cm}
\texttt{True, 1, False}
\end{solutionbox}

\question[3] Was erscheint auf der Konsole?
\begin{lstlisting}
u = {4,7,9,10}
v = {4,8,9,11}
print(u | v)
print(u & v)
print(u - v)
print(v - u)
\end{lstlisting}
\begin{solutionbox}{4cm}
\begin{lstlisting}
{4, 7, 8, 9, 10, 11}
{9, 4}
{10, 7}
{8, 11}
\end{lstlisting}
\end{solutionbox}

\question[2] Was erscheint auf der Konsole?
\begin{lstlisting}
m = {'a':5, 'c':3, 'd':3, 'e':1}
a = sorted(list(set(m.values())))
print(a)
\end{lstlisting}
\begin{solutionbox}{2cm}
\begin{lstlisting}
[1, 3, 5]
\end{lstlisting}
\end{solutionbox}

\question[1] Was erscheint auf der Konsole?
\begin{lstlisting}
m = {'a':3, 'c':3, 'd':3, 'e':3}
a = len(set(m.items()))
print(a)
\end{lstlisting}
\begin{solutionbox}{2cm}
\begin{lstlisting}
4
\end{lstlisting}
\end{solutionbox}

\newpage
\question[4] Implementiere die Funktion \texttt{summeKleine}
\begin{lstlisting}
def summeKleine(g):
    '''
    g: dictionary, das einem Buchstaben eine ganze Zahl zuordnet
    returns: Summe aller values, die einem kleinen Buchstaben zugeordnet sind

    z.B:
    >>> G = {'a':3, 'c':7, 'D':8, 'e':10, 'B':7}
    >>> summeKleine(G)
    20
    '''
\end{lstlisting}
\begin{solutionbox}{7cm}
\begin{lstlisting}
def summeKleine(g):
    summe = 0
    for x in g:
        if x.islower():
            summe+=g[x]
    return summe
\end{lstlisting}
\end{solutionbox}

\question[4] Implementiere die Funktion.
\begin{lstlisting}
def anzahlZweier(s):
    '''
    s: String
    returns: Anzahl der verschiedenen Teilstrings mit Länge 2 in s

    Beispiel:
    >>> s = 'abababc'
    >>> anzahlZweier(s)
    3
    '''
\end{lstlisting}
\begin{solutionbox}{6cm}
\begin{lstlisting}
def anzahlZweier(s):
    tmp = set()
    for i in range(len(s)-1):
        tmp.add(s[i:i+2])
    return len(tmp)

\end{lstlisting}
\end{solutionbox}





% -------------------------------------------------
\end{questions}
\begin{center}
%\pointtable[h][questions]
\end{center}

\end{document}
