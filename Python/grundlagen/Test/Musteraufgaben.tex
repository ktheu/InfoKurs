\documentclass[addpoints]{exam}
\usepackage[utf8x]{inputenc}
\usepackage[ngerman]{babel}
\usepackage{listings}
\usepackage{babel}
\usepackage[top=1.5cm,bottom=0.5cm,headsep=0.5cm,headheight=3cm,%
left=1.5cm,right=1.5cm]{geometry}

\usepackage[T1]{fontenc}
\usepackage{booktabs} % schöne Tabellen
\usepackage{graphicx}
\usepackage{csquotes} % Anführungszeichen
\usepackage{paralist} % kompakte Aufzählungen
\usepackage{amsmath,textcomp,tikz} %diverses
\usepackage{eso-pic} % Bilder im Hintergrund
\usepackage{mdframed} % Boxen
\usepackage{multirow}


\newmdenv[linecolor=black,backgroundcolor=gray!15,
frametitle={Punktverteilung},leftmargin=1cm,
rightmargin=1cm]{infobox}

\lstset{language=Python, tabsize=4, basicstyle=\footnotesize, showstringspaces=false, mathescape=true}
\lstset{literate=%
  {Ö}{{\"O}}1
  {Ä}{{\"A}}1
  {Ü}{{\"U}}1
  {ß}{{\ss}}1
  {ü}{{\"u}}1
  {ä}{{\"a}}1
  {ö}{{\"o}}1
}
\begin{document}
\pointpoints{Punkt}{Punkte}
\bonuspointpoints{Bonuspunkt}{Bonuspunkte}
\renewcommand{\solutiontitle}{\noindent\textbf{Lösung:}%
\enspace}

\chqword{Frage}
\chpgword{Seite}
\chpword{Punkte}
\chbpword{Bonus Punkte}
\chsword{Erreicht}
\chtword{Gesamt}
\hpword{Punkte:} % Punktetabelle
\hsword{Ergebnis:}
\hqword{Aufgabe:}
\htword{Summe:}
\cellwidth{1.5em}
%\begin{center}
%\fbox{\fbox{\parbox{5.5in}{\centering
%Informatik-Klausur}}}
%\end{center}
%
%\vspace{5mm}
%
%\makebox[\textwidth]{Name:\enspace\hrulefill}
\pagestyle{headandfoot}
\runningheadrule

\newcommand\Vtextvisiblespace[1][.3em]{%
  \mbox{\kern.06em\vrule height.3ex}%
  \vbox{\hrule width#1}%
  \hbox{\vrule height.3ex}}

\newcommand{\klaubez}{Aufgaben zu Variablen, Typen, arithmetische und boolesche Operatoren}
\firstpageheader{Informatik }{\klaubez} {\thepage /\numpages}
\runningheader{Informatik }{\klaubez} {\thepage /\numpages}
\newcommand{\pfad}{c:/Users/khthe/Dropbox/Informatik/KursV2/Python}
%-------------------------------------------------------------------
%\printanswers
%-------------------------------------------------------------------

\begin{questions}
\question[3]
Gib jeweils den Typ des Ausdrucks an. \\
Zur Auswahl stehen: int, float, bool, str, NoneType. Wenn nichts von allen zutrifft, notiere
ein Minuszeichen -.
\begin{lstlisting}
a. 3.13    b. -34    c. True    d. None   e. 3,17    f. '3.17'
\end{lstlisting}
\begin{solutionbox}{2cm}
\begin{lstlisting}
a. float, b. int, c. bool, d. NoneType, e. -, f. str
\end{lstlisting}
\end{solutionbox}

\question[3]
Gib jeweils den Typ des Ausdrucks an. \\
Zur Auswahl stehen: int, float, bool, str, NoneType. Wenn nichts von allen zutrifft, notiere
ein Minuszeichen -.
\begin{lstlisting}
a. 5.0    b. false    c. NoneType    d. 12.3   e. '-1'    f. 42
\end{lstlisting}
\begin{solutionbox}{2cm}
\begin{lstlisting}
a. float, b. -, c. -, d. float, e. str, f. int
\end{lstlisting}
\end{solutionbox}

\question[3]
Zu was werten sich die Ausdrücke aus? Wenn die Auswertung zu einem Fehler führt, notiere das
Wort 'error'. Gib bei Gleitkommazahlen höchstens 4 Stellen nach dem Komma an.

\begin{lstlisting}
a. 2 * 3.0    b. $- - 4$     c. 10/3    d. 10//4   e. 2 + 3 * 4    f. 2**3 + 1
\end{lstlisting}
\begin{solutionbox}{2cm}
\begin{lstlisting}
a. 6.0, b. 4, c. 3.3333, d. 2, e. 14, f. 9
\end{lstlisting}
\end{solutionbox}

\question[3]
Zu was werten sich die Ausdrücke aus? Wenn die Auswertung zu einem Fehler führt, notiere das
Wort 'error'. Gib bei Gleitkommazahlen höchstens 4 Stellen nach dem Komma an.

\begin{lstlisting}
a. 2.0 + 5    b. -4/1    c. 12//3    d. 12/3   e. 12%3    f. 12/0
\end{lstlisting}
\begin{solutionbox}{2cm}
\begin{lstlisting}
a. 7.0, b. -4.0, c. 4, d. 4.0, e. 0, f. error
\end{lstlisting}
\end{solutionbox}

\question[3]
Zu was werten sich die Ausdrücke aus? Wenn die Auswertung zu einem Fehler führt, notiere das
Wort 'error'. Gib bei Gleitkommazahlen höchstens 4 Stellen nach dem Komma an.
\begin{lstlisting}
a. 5 + 7 * 2 ** 3 - 2 * 2 ** 3    b. 12 // 3 + 4 % 3 * 2 ** 2
c. 14 % (4 // 3) ** 2
\end{lstlisting}
\begin{solutionbox}{2cm}
\begin{lstlisting}
a. 45, b. 8, c. 0
\end{lstlisting}
\end{solutionbox}

\question[3]
Zu was werten sich die Ausdrücke aus? Wenn die Auswertung zu einem Fehler führt, notiere das
Wort 'error'. Gib bei Gleitkommazahlen höchstens 4 Stellen nach dem Komma an.
\begin{lstlisting}
a. 4 + 2 ** 3 ** 2 // 3    b. 15 % 8 // 2 * 3
c. 4 + 4 * (4 // 4 ** 4)
\end{lstlisting}
\begin{solutionbox}{2cm}
\begin{lstlisting}
a. 174, b. 9, c. 4
\end{lstlisting}
\end{solutionbox}

\question[3]
Die folgenden Anweisungen werden nacheinander in der shell eingegeben. Um auf die Anweisungen Bezug
zu nehmen, sind sie mit Nummern versehen. Die Ausgaben sind nicht notiert.
Notiere zu jeder Eingabe, welche Ausgabe
erscheint. Wenn keine Ausgabe erfolgt, notiere
ein Minuszeichen. Wenn die Ausgabe zu einem Fehler führt, notiere 'error'.

\begin{lstlisting}
(1) >>> a = 3
(2) >>> a + 2.0
(3) >>> a = a + 1.0
(4) >>> a
(5) >>> a = 3
(6) >>> b
\end{lstlisting}
\begin{solutionbox}{5cm}
\begin{lstlisting}
(1) -
(2) 5.0
(3) -
(4) 4.0 
(5) -
(6) error
\end{lstlisting}
\end{solutionbox}

\question[3]
Die folgenden Anweisungen werden nacheinander in der shell eingegeben. Um auf die Anweisungen Bezug
zu nehmen, sind sie mit Nummern versehen. Die Ausgaben sind nicht notiert.
Notiere zu jeder Eingabe, welche Ausgabe
erscheint. Wenn keine Ausgabe erfolgt, notiere
ein Minuszeichen. Wenn die Ausgabe zu einem Fehler führt, notiere 'error'.

\begin{lstlisting}
(1) >>> c, d = 3, 1.0
(2) >>> e = c + d
(3) >>> e + 2
(4) >>> e
(5) >>> d = 'a'
(6) >>> d*=c
(7) >>> d
\end{lstlisting}
\begin{solutionbox}{5cm}
\begin{lstlisting}
(1) -
(2) -
(3) 6.0
(4) 4.0
(5) -
(6) -
(7) 'aaa'
\end{lstlisting}
\end{solutionbox}

\question[2]
Schreibe für die beiden Zeilen Kurzformen (erweiterte Zuweisungen).
\begin{lstlisting}
k = k + 5
j = j * k
\end{lstlisting}
\begin{solutionbox}{2cm}
\begin{lstlisting}
k+=5
j*=k
\end{lstlisting}
\end{solutionbox}

\question[2]
Schreibe für die beiden Zeilen Kurzformen (erweiterte Zuweisungen).
\begin{lstlisting}
m = m % 2
n = n - m
\end{lstlisting}
\begin{solutionbox}{2cm}
\begin{lstlisting}
m%=2
n-=m
\end{lstlisting}
\end{solutionbox}

\question[3]
Zu was werten sich die Ausdrücke aus? Setze Strings in einfache Hochkommata.
Wenn die Auswertung zu einem Fehler führt, notiere das
Wort 'error'.

\begin{lstlisting}
a. 6 + '7'   b. '6' + '7'  c. 6 * '7'  d. '6' * 7   e. '6' * '7'  f. 6 * 7
\end{lstlisting}
\begin{solutionbox}{2cm}
\begin{lstlisting}
a. error, b. '67' c. '777777', d. '6666666' e. error, f. 42
\end{lstlisting}
\end{solutionbox}

\question[2]
Zu was werten sich die Ausdrücke aus? Setze Strings in einfache Hochkommata.
Wenn die Auswertung zu einem Fehler führt, notiere das
Wort 'error'.

\begin{lstlisting}
a. 'None' * 2   b. 'abc' + (2 * 'ab')  c. '3' * 4  d. 3 * '4'
\end{lstlisting}
\begin{solutionbox}{2cm}
\begin{lstlisting}
a. 'NoneNone', b. 'abcabab', c. '3333',  d. '444'
\end{lstlisting}
\end{solutionbox}

\question[3]
Zu was werten sich die Ausdrücke aus? Setze einen String in einfache Hochkommata.
Wenn die Auswertung zu einem Fehler führt, notiere das
Wort 'error'.

\begin{lstlisting}
a. float(5)    b. str(3)    c. bool(2)    d. int(3.2)   e. float('5')    f. int('5.0')
\end{lstlisting}
\begin{solutionbox}{2cm}
\begin{lstlisting}
a. 5.0, b. '3', c. True, d. 3, e. 5.0, f. error
\end{lstlisting}
\end{solutionbox}

\question[3]
Zu was werten sich die Ausdrücke aus? Setze einen String in einfache Hochkommata.
Wenn die Auswertung zu einem Fehler führt, notiere das
Wort 'error'.

\begin{lstlisting}
a. bool(5)    b. str('5')    c. bool('')    d. bool('False')   e. float('5.0')    f. int('-17')
\end{lstlisting}
\begin{solutionbox}{2cm}
\begin{lstlisting}
a. True, b. '5', c. False, d. True, e. 5.0, f. -17
\end{lstlisting}
\end{solutionbox}

\question[2]
Es gelte: a, b, c = 2, 3, 1  \\
Zu was werten sich die folgenden Ausdrücke aus?
\begin{lstlisting}
a. 2*a <= c+1 or a <= c and a > b
b. not a >= 2*c or b+1 $==$ 2*a
\end{lstlisting}

\begin{solutionbox}{1cm}
a. False ~~ b. True
\end{solutionbox}

\question[2]
Es gelte: a, b, c = 3, 9, 2\\
Zu was werten sich die folgenden Ausdrücke aus?
\begin{lstlisting}
a. c % a > b / c and not a <= b - c
b. not a * c > b and a !=b % c
\end{lstlisting}

\begin{solutionbox}{1cm}
a. False ~~ b. True
\end{solutionbox}



% -------------------------------------------------
\end{questions}
\begin{center}
%\pointtable[h][questions]
\end{center}

\end{document}
