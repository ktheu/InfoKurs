\question[3]
Die folgenden Anweisungen werden nacheinander in der shell eingegeben. Um auf die Anweisungen Bezug
zu nehmen, sind sie mit Nummern versehen. Die Ausgaben sind nicht notiert.
Notiere zu jeder Eingabe, welche Ausgabe
erscheint. Wenn keine Ausgabe erfolgt, notiere
ein Minuszeichen. Wenn die Ausgabe zu einem Fehler führt, notiere 'error'.

\begin{lstlisting}
(1) >>> c, d = 3, 1.0
(2) >>> e = c + d
(3) >>> e + 2
(4) >>> e
(5) >>> d = 'a'
(6) >>> d*=c
(7) >>> d
\end{lstlisting}
\begin{solutionbox}{5cm}
\begin{lstlisting}
(1) -
(2) -
(3) 6.0
(4) 4.0
(5) -
(6) -
(7) 'aaa'
\end{lstlisting}
\end{solutionbox}
