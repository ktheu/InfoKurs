\documentclass[addpoints]{exam}
\usepackage[utf8x]{inputenc}
\usepackage[ngerman]{babel}
\usepackage{listings}
\usepackage{babel}
\usepackage[top=1.5cm,bottom=0.5cm,headsep=0.5cm,headheight=3cm,%
left=1.5cm,right=1.5cm]{geometry}

\usepackage[T1]{fontenc}
\usepackage{booktabs} % schöne Tabellen
\usepackage{graphicx}
\usepackage{csquotes} % Anführungszeichen
\usepackage{paralist} % kompakte Aufzählungen
\usepackage{amsmath,textcomp,tikz} %diverses
\usepackage{eso-pic} % Bilder im Hintergrund
\usepackage{mdframed} % Boxen
\usepackage{multirow}


\newmdenv[linecolor=black,backgroundcolor=gray!15,
frametitle={Punktverteilung},leftmargin=1cm,
rightmargin=1cm]{infobox}

\lstset{language=Python, tabsize=4, basicstyle=\footnotesize, showstringspaces=false, mathescape=true,commentstyle=\fontfamily{fve}  }
\lstset{literate=%
  {Ö}{{\"O}}1
  {Ä}{{\"A}}1
  {Ü}{{\"U}}1
  {ß}{{\ss}}1
  {ü}{{\"u}}1
  {ä}{{\"a}}1
  {ö}{{\"o}}1
}
\begin{document}
\pointpoints{Punkt}{Punkte}
\bonuspointpoints{Bonuspunkt}{Bonuspunkte}
\renewcommand{\solutiontitle}{\noindent\textbf{Lösung:}%
\enspace}

\chqword{Frage}
\chpgword{Seite}
\chpword{Punkte}
\chbpword{Bonus Punkte}
\chsword{Erreicht}
\chtword{Gesamt}
\hpword{Punkte:} % Punktetabelle
\hsword{Ergebnis:}
\hqword{Aufgabe:}
\htword{Summe:}
\cellwidth{1.5em}
%\begin{center}
%\fbox{\fbox{\parbox{5.5in}{\centering
%Informatik-Klausur}}}
%\end{center}
%
%\vspace{5mm}
%
%\makebox[\textwidth]{Name:\enspace\hrulefill}
\pagestyle{headandfoot}
\runningheadrule

\newcommand\Vtextvisiblespace[1][.3em]{%
  \mbox{\kern.06em\vrule height.3ex}%
  \vbox{\hrule width#1}%
  \hbox{\vrule height.3ex}}

\newcommand{\klaubez}{Aufgaben zu Dictionaries}
\firstpageheader{Informatik }{\klaubez} {\thepage /\numpages}
\runningheader{Informatik }{\klaubez} {\thepage /\numpages}
\newcommand{\pfad}{c:/Users/khthe/Dropbox/Informatik/KursV2/Python}
%-------------------------------------------------------------------
%\printanswers
%-------------------------------------------------------------------

\begin{questions}

\question[6] Was erscheint auf der Konsole?
 Notiere die einzelnen Ausgaben aus Platzgründen horizontal mit Komma getrennt,
 schreibe \texttt{Leerzeile} für eine Leerzeile.
\begin{lstlisting}
m = {'a':'gb', 'b':'g','g':'b'}
print(m['a'])
print('a' in m )
print('gb' in m)
print('b' in m['g'])
print(m['b'] in m)
print(len(m))
\end{lstlisting}
\begin{solutionbox}{2cm}
\texttt{gb, True, False, True, True, 3}
\end{solutionbox}

%%dictionaries A2 ---------------------------
\question[3] Was erscheint auf der Konsole?  
Notiere die einzelnen Ausgaben aus Platzgründen horizontal mit Komma getrennt,
schreibe \texttt{Leerzeile} für eine Leerzeile.
\begin{lstlisting}
m = {'a':'c','b':'gh','g':'bghfa','m':'b'}
print(len(m))
print(m['b'])
print('c' in m)
print('m' in m)
print(m['m'] in m)
print(m['b'] in m['g'])
\end{lstlisting}
\begin{solutionbox}{1cm}
\texttt{4, gh, False, True, True, True}
\end{solutionbox}

\question[3] Was erscheint auf der Konsole?
Notiere die einzelnen Ausgaben aus Platzgründen horizontal mit Komma getrennt,
schreibe \texttt{Leerzeile} für eine Leerzeile.
\begin{lstlisting}
m = {'a':'d','b':'a','g':'abba','k':'bob'}
print(len(m))
print(m['g'])
print(m['b'] in m['g'])
print('bob' in m)
print(m['b'] in m.values())
print(m['a'] in m.keys())
\end{lstlisting}
\begin{solutionbox}{1cm}
\texttt{4, abba, True, False, True, False}
\end{solutionbox}

%\question[6] Was erscheint auf der Konsole?  
Notiere die einzelnen Ausgaben aus Platzgründen horizontal mit Komma getrennt,
schreibe \texttt{Leerzeile} für eine Leerzeile.
\begin{lstlisting}
m = {'a':'abba','b':'bob','d':'b'}
print(len(m))
m['a'] = 'bab'
print(m['b'])
print(m['a'])
print('bob' in m)
print('bob' in m.keys())
print(m['d'] in m['a'])
\end{lstlisting}
\begin{solutionbox}{1cm}
\texttt{3, bob, bab, False, False, True}
\end{solutionbox}

\question[6] G sei wie abgebildet ein dictionary, bei dem jedem key
ein weiteres dictionary zugeordnet ist. Schreibe Code, der alle Werte
des dictionaries ausgibt, das in G dem Schlüssel 'e' zugeordnet ist.
\begin{lstlisting}
G = {
'a': {'f':8, 'b':10},
'b': {'d':2},
'c': {'b':1},
'd': {'c':-2},
'e': {'d':-1,'b':-4},
'f': {'e':1}
}
\end{lstlisting}
\begin{solutionbox}{2cm}
\begin{lstlisting}
for x in G['e'].values():
    print(x)
\end{lstlisting}    
\end{solutionbox}

\question[2] Schreibe eine Schleife, die die keys des dictionaries ausgibt.
\begin{lstlisting}
mu4 = {1:5, 2:7, 3:9, 4:9}
\end{lstlisting}
\begin{solutionbox}{2cm}
\begin{lstlisting}
for x in mu4:
    print(x)
\end{lstlisting}
\end{solutionbox}

%\question[2] Schreibe eine Schleife, die die values des dictionaries ausgibt.

\texttt{hugo = {1:5, 2:7, 3:9, 4:9}}

\begin{solutionbox}{2cm}
\begin{lstlisting}
for x in hugo.values():
    print(x)
\end{lstlisting}
\end{solutionbox}

%\question[2] Schreibe eine Schleife, die die key-value Paare des
dictionaries als Tupel ausgibt.

\texttt{anton = {1:5, 2:7, 3:9, 4:9}}

\begin{solutionbox}{2cm}
\begin{lstlisting}
for x in anton.items():
    print(x)
\end{lstlisting}
\end{solutionbox}

\question[3] Implementiere die Funktion \texttt{bereich}.
\begin{lstlisting}
def bereich(u,v,x1,x2):
    '''
    u, v: sets mit ganzen Zahlen
    x1, x2: ganze Zahlen
    returns: set mit allen Zahlen k aus u oder v, für die
        x1 <= k <= x2 gilt.

    Beispiel:
    >>> u = {-1,3,5,17,9}
    >>> v = {2,14,6,28,10}
    >>> bereich(u,v,5,8)
    {5, 6}
    '''
\end{lstlisting}
\begin{solutionbox}{6cm}
\begin{lstlisting}
def bereich(u,v,x1,x2):
    tmp = set()
    for x in (u | v):
        if x1 <= x <= x2:
            tmp.add(x)
    return tmp
\end{lstlisting}
\end{solutionbox}

%\question[4] Implementiere die Funktion \texttt{summeKleine}
\begin{lstlisting}
def summeKleine(g):
    '''
    g: dictionary, das einem Buchstaben eine ganze Zahl zuordnet
    returns: Summe aller values, die einem kleinen Buchstaben zugeordnet sind

    z.B:
    >>> G = {'a':3, 'c':7, 'D':8, 'e':10, 'B':7}
    >>> summeKleine(G)
    20
    '''
\end{lstlisting}
\begin{solutionbox}{7cm}
\begin{lstlisting}
def summeKleine(g):
    summe = 0
    for x in g:
        if x.islower():
            summe+=g[x]
    return summe
\end{lstlisting}
\end{solutionbox}




% -------------------------------------------------
\end{questions}
\begin{center}
%\pointtable[h][questions]
\end{center}

\end{document}
