\question[5]  Implementiere die Funktion.

\begin{lstlisting}
def groessterRest(s):
    '''
    s: String mit Ziffern, mindestens Länge 2
    returns: int, Index an dem der zweistellige Teilstring beginnt
        der den größten Rest bei der Division durch 7 ergibt.

    Beispiel:
    >>> groessterRest('532343456')
    3
    denn bei Index 3 beginnt die zweistellige Zahl 34,
       die bei Division durch 7 den Rest 6 hat.
    '''
\end{lstlisting}
\begin{solutionbox}{10cm}
\begin{lstlisting}
def groessterRest(s):
    king = 0
    kingRest = int(s[:2]) % 7

    for i in range(1,len(s)-1):
        teil = int(s[i:i+2])
        if teil % 7 > kingRest:
            king = i
            kingRest = teil % 7
    return king
\end{lstlisting}
\end{solutionbox}
