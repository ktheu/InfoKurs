\question[5] Die Quersumme einer Zahl ist die Summe ihrer Ziffern.
Implementiere die Funktion maxQuer.

\begin{lstlisting}
def maxQuer(a,b,c,d):
    '''
    a,b,c,d: positive ganze Zahlen
    returns: die Zahl mit der größten Quersumme.
    '''
\end{lstlisting}
\begin{solutionbox}{10cm}
\begin{lstlisting}
def quersumme(k):
    summe = 0
    for c in str(k):
        summe += int(c)
    return summe

def maxQuer(a,b,c,d):
    '''
    a,b,c,d: positive ganze Zahlen
    returns: die Zahl mit der größten Quersumme.
    '''
    king = a
    kingWert = quersumme(a)
    if quersumme(b) > kingWert:
        kingWert = quersumme(b)
        king = b
    if quersumme(c) > kingWert:
        kingWert = quersumme(c)
        king = c
    if quersumme(d) > kingWert:
        kingWert = quersumme(d)
        king = d
    return king
\end{lstlisting}
\end{solutionbox}
