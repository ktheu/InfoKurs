\documentclass[addpoints]{exam}
\usepackage[utf8x]{inputenc}
\usepackage[ngerman]{babel}
\usepackage{listings}
\usepackage{babel}
\usepackage[top=1.5cm,bottom=0.5cm,headsep=0.5cm,headheight=3cm,%
left=1.5cm,right=1.5cm]{geometry}

\usepackage[T1]{fontenc}
\usepackage{booktabs} % schöne Tabellen
\usepackage{graphicx}
\usepackage{csquotes} % Anführungszeichen
\usepackage{paralist} % kompakte Aufzählungen
\usepackage{amsmath,textcomp,tikz} %diverses
\usepackage{eso-pic} % Bilder im Hintergrund
\usepackage{mdframed} % Boxen
\usepackage{multirow}


\newmdenv[linecolor=black,backgroundcolor=gray!15,
frametitle={Punktverteilung},leftmargin=1cm,
rightmargin=1cm]{infobox}

\lstset{language=Python, tabsize=4, basicstyle=\footnotesize, showstringspaces=false, mathescape=true}
\lstset{literate=%
  {Ö}{{\"O}}1
  {Ä}{{\"A}}1
  {Ü}{{\"U}}1
  {ß}{{\ss}}1
  {ü}{{\"u}}1
  {ä}{{\"a}}1
  {ö}{{\"o}}1
}
\begin{document}
\pointpoints{Punkt}{Punkte}
\bonuspointpoints{Bonuspunkt}{Bonuspunkte}
\renewcommand{\solutiontitle}{\noindent\textbf{Lösung:}%
\enspace}

\chqword{Frage}
\chpgword{Seite}
\chpword{Punkte}
\chbpword{Bonus Punkte}
\chsword{Erreicht}
\chtword{Gesamt}
\hpword{Punkte:} % Punktetabelle
\hsword{Ergebnis:}
\hqword{Aufgabe:}
\htword{Summe:}
\cellwidth{1.5em}
%\begin{center}
%\fbox{\fbox{\parbox{5.5in}{\centering
%Informatik-Klausur}}}
%\end{center}
%
%\vspace{5mm}
%
%\makebox[\textwidth]{Name:\enspace\hrulefill}
\pagestyle{headandfoot}
\runningheadrule

\newcommand\Vtextvisiblespace[1][.3em]{%
  \mbox{\kern.06em\vrule height.3ex}%
  \vbox{\hrule width#1}%
  \hbox{\vrule height.3ex}}

\newcommand{\klaubez}{Aufgaben zu Stringvergleich, Königssuche}
\firstpageheader{Informatik }{\klaubez} {\thepage /\numpages}
\runningheader{Informatik }{\klaubez} {\thepage /\numpages}
\newcommand{\pfad}{c:/Users/khthe/Dropbox/Informatik/KursV2/Python}
%-------------------------------------------------------------------
%\printanswers
%-------------------------------------------------------------------

\begin{questions}
\question[4] Implementiere die Funktion.

\begin{lstlisting}
def mehrKleine(s):
    '''
    s: String
    returns: True, wenn in s mehr Zeichen a-z als Zeichen A-Z vorkommen
    '''
\end{lstlisting}
\begin{solutionbox}{8cm}
\begin{lstlisting}
def mehrKleine(s):
    summeKlein = 0
    summeGross = 0
    for c in s:
        if 'a' <= c <= 'z': summeKlein += 1
        elif 'A' <= c <= 'Z': summeGross += 1
    return summeKlein > summeGross
\end{lstlisting}
\end{solutionbox}

\question[3] Implementiere die Funktion.

\begin{lstlisting}
def anzahlZiffern(s):
    '''
    s: String
    returns: int, Anzahl der Ziffern in s
    '''
\end{lstlisting}
\begin{solutionbox}{8cm}
\begin{lstlisting}
def anzahlZiffern(s):
    zaehl = 0
    for c in s:
        if '0' <= c <= '9': zaehl += 1
    return zaehl
\end{lstlisting}
\end{solutionbox}

\question[3] Implementiere die Funktion. Die eingebauten Python-Funktionen min
und max dürfen nicht verwendet werden.

\begin{lstlisting}
def minZahl(a,b,c,d,e):
    '''
    a,b,c,d,e: Zahlen (ints oder floats)
    returns: das Minimum der Zahlen.
    '''
\end{lstlisting}
\begin{solutionbox}{8cm}
\begin{lstlisting}
def minZahl(a,b,c,d,e):
    king = a
    if b < king: king = b
    if c < king: king = c
    if d < king: king = d
    if e < king: king = e
    return king
\end{lstlisting}
\end{solutionbox}

\question[5] Die Quersumme einer Zahl ist die Summe ihrer Ziffern.
Implementiere die Funktion maxQuer.

\begin{lstlisting}
def maxQuer(a,b,c,d):
    '''
    a,b,c,d: positive ganze Zahlen
    returns: die Zahl mit der größten Quersumme.
    '''
\end{lstlisting}
\begin{solutionbox}{10cm}
\begin{lstlisting}
def quersumme(k):
    summe = 0
    for c in str(k):
        summe += int(c)
    return summe

def maxQuer(a,b,c,d):
    '''
    a,b,c,d: positive ganze Zahlen
    returns: die Zahl mit der größten Quersumme.
    '''
    king = a
    kingWert = quersumme(a)
    if quersumme(b) > kingWert:
        kingWert = quersumme(b)
        king = b
    if quersumme(c) > kingWert:
        kingWert = quersumme(c)
        king = c
    if quersumme(d) > kingWert:
        kingWert = quersumme(d)
        king = d
    return king
\end{lstlisting}
\end{solutionbox}

\question[5]  Implementiere die Funktion.

\begin{lstlisting}
def groessterRest(s):
    '''
    s: String mit Ziffern, mindestens Länge 2
    returns: int, Index an dem der zweistellige Teilstring beginnt
        der den größten Rest bei der Division durch 7 ergibt.

    Beispiel:
    >>> groessterRest('532343456')
    3
    denn bei Index 3 beginnt die zweistellige Zahl 34,
       die bei Division durch 7 den Rest 6 hat.
    '''
\end{lstlisting}
\begin{solutionbox}{10cm}
\begin{lstlisting}
def groessterRest(s):
    king = 0
    kingRest = int(s[:2]) % 7

    for i in range(1,len(s)-1):
        teil = int(s[i:i+2])
        if teil % 7 > kingRest:
            king = i
            kingRest = teil % 7
    return king
\end{lstlisting}
\end{solutionbox}






% -------------------------------------------------
\end{questions}
\begin{center}
%\pointtable[h][questions]
\end{center}

\end{document}
