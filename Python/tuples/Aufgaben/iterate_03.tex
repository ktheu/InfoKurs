\question[3]
Schreibe eine Funktion
\texttt{anzahlKleineA}, der ein Tupel t mit Strings übergeben wird und 
die zurückgibt, wieviele Zeichen 'a' insgesamt in den Elementen von t vorkommen.


Beispiel:
\begin{lstlisting}
>>> anzahlKleineA(('aab', 'cad', 'eaaf'))
5
\end{lstlisting}
\begin{solutionbox}{4cm}
\begin{lstlisting}
def anzahlKleineA(t):
    summe = 0
    for s in t:
        summe += s.count('a')
    return summe
\end{lstlisting}
\end{solutionbox}
