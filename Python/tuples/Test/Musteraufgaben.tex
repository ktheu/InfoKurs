\documentclass[addpoints]{exam}
\usepackage[utf8x]{inputenc}
\usepackage[ngerman]{babel}
\usepackage{listings}
\usepackage{babel}
\usepackage[top=1.5cm,bottom=0.5cm,headsep=0.5cm,headheight=3cm,%
left=1.5cm,right=1.5cm]{geometry}

\usepackage[T1]{fontenc}
\usepackage{booktabs} % schöne Tabellen
\usepackage{graphicx}
\usepackage{csquotes} % Anführungszeichen
\usepackage{paralist} % kompakte Aufzählungen
\usepackage{amsmath,textcomp,tikz} %diverses
\usepackage{eso-pic} % Bilder im Hintergrund
\usepackage{mdframed} % Boxen
\usepackage{multirow}


\newmdenv[linecolor=black,backgroundcolor=gray!15,
frametitle={Punktverteilung},leftmargin=1cm,
rightmargin=1cm]{infobox}

\lstset{language=Python, tabsize=4, basicstyle=\footnotesize, showstringspaces=false, mathescape=true,commentstyle=\fontfamily{fve}  }
\lstset{literate=%
  {Ö}{{\"O}}1
  {Ä}{{\"A}}1
  {Ü}{{\"U}}1
  {ß}{{\ss}}1
  {ü}{{\"u}}1
  {ä}{{\"a}}1
  {ö}{{\"o}}1
}
\begin{document}
\pointpoints{Punkt}{Punkte}
\bonuspointpoints{Bonuspunkt}{Bonuspunkte}
\renewcommand{\solutiontitle}{\noindent\textbf{Lösung:}%
\enspace}

\chqword{Frage}
\chpgword{Seite}
\chpword{Punkte}
\chbpword{Bonus Punkte}
\chsword{Erreicht}
\chtword{Gesamt}
\hpword{Punkte:} % Punktetabelle
\hsword{Ergebnis:}
\hqword{Aufgabe:}
\htword{Summe:}
\cellwidth{1.5em}
%\begin{center}
%\fbox{\fbox{\parbox{5.5in}{\centering
%Informatik-Klausur}}}
%\end{center}
%
%\vspace{5mm}
%
%\makebox[\textwidth]{Name:\enspace\hrulefill}
\pagestyle{headandfoot}
\runningheadrule

\newcommand\Vtextvisiblespace[1][.3em]{%
  \mbox{\kern.06em\vrule height.3ex}%
  \vbox{\hrule width#1}%
  \hbox{\vrule height.3ex}}

\newcommand{\klaubez}{Aufgaben zu Tupeln}
\firstpageheader{Informatik }{\klaubez} {\thepage /\numpages}
\runningheader{Informatik }{\klaubez} {\thepage /\numpages}
\newcommand{\pfad}{c:/Users/khthe/Dropbox/Informatik/KursV2/Python}
%-------------------------------------------------------------------
%\printanswers
%-------------------------------------------------------------------

\begin{questions}
\question[3]
a. Weise der Variablen a ein leeres Tupel zu \\
b. Weise der Variablen b ein Tupel mit dem Element 'B' zu. \\
c. Weise der Variablen c ein Tupel mit den booleschen Werten True, True, False zu.
\begin{solutionbox}{2cm}
\begin{lstlisting}
a = ()
b = ('B',)
c = (True, True, False)
\end{lstlisting}
\end{solutionbox}

\question[3]
Was erscheint auf der Konsole?
\begin{lstlisting}
a = [4,12,5,20,9]
b = [max(a),min(a),len(a)]
c = [7 in a, 10 not in a]
d = a + b + c
print(d)
\end{lstlisting}
\begin{solutionbox}{1cm}
\texttt{[4, 12, 5, 20, 9, 20, 4, 5, False, True]}
\end{solutionbox}

\question[2]
Was wird an der Konsole ausgegeben?
\begin{lstlisting}
a = (1,4)
b = a * 2
c = b + (5,4,7)
d = c[2:5] + b[3:4]
print(d)
\end{lstlisting}
\begin{solutionbox}{2cm}
\begin{lstlisting}
(1, 4, 5, 4)
\end{lstlisting}
\end{solutionbox}

\question[2]
Schreibe eine Funktion \texttt{minmax}, der ein nicht leeres Tupel aus Zahlen
übergeben wird und die das Mininum und Maximum der Zahlen zurückgibt.
\begin{solutionbox}{3cm}
\begin{lstlisting}
def minmax(t):
    return min(t),max(t)
\end{lstlisting}
\end{solutionbox}

\question[4]
Implementiere die Funktion \texttt{maxchar}
\begin{lstlisting}
def maxchar(s):
    '''
    s: nicht leerer String
    returns: Tupel (c,k)
       c: Zeichen, das am häufigsten in s vorkommt, bei mehreren Zeichen das erste.
       k: Anzahl der Vorkommen von c

    Beispiel:
    >>> s = 'abbaacccccaa'
    >>> maxchar(s)
    ('a', 5)
    '''
\end{lstlisting}
\begin{solutionbox}{9cm}
\begin{lstlisting}
def maxchar(s):
    maxZeichen = s[0]
    maxCount = s.count(maxZeichen)
    for c in s:
        if s.count(c) > maxCount:
            maxZeichen = c
            maxCount = s.count(c)
    return maxZeichen, maxCount
\end{lstlisting}
\end{solutionbox}

\question[2]
Gegeben ein Tupel \texttt{hugo} aus Strings. Gehe durch das Tupel und gib
die Längen der Strings aus.
\begin{solutionbox}{4cm}
\begin{lstlisting}
for s in hugo:
    print(len(s))
\end{lstlisting}
\end{solutionbox}






% -------------------------------------------------
\end{questions}
\begin{center}
%\pointtable[h][questions]
\end{center}

\end{document}
