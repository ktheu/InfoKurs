\documentclass[addpoints]{exam}
\usepackage[utf8x]{inputenc}
\usepackage[ngerman]{babel}
\usepackage{listings}
\usepackage{babel}
\usepackage[top=1.5cm,bottom=0.5cm,headsep=0.5cm,headheight=3cm,%
left=1.5cm,right=1.5cm]{geometry}

\usepackage[T1]{fontenc}
\usepackage{booktabs} % schöne Tabellen
\usepackage{graphicx}
\usepackage{csquotes} % Anführungszeichen
\usepackage{paralist} % kompakte Aufzählungen
\usepackage{amsmath,textcomp,tikz} %diverses
\usepackage{eso-pic} % Bilder im Hintergrund
\usepackage{mdframed} % Boxen
\usepackage{multirow}


\newmdenv[linecolor=black,backgroundcolor=gray!15,
frametitle={Punktverteilung},leftmargin=1cm,
rightmargin=1cm]{infobox}

\lstset{language=Python, tabsize=4, basicstyle=\footnotesize, showstringspaces=false, mathescape=true}
\lstset{literate=%
  {Ö}{{\"O}}1
  {Ä}{{\"A}}1
  {Ü}{{\"U}}1
  {ß}{{\ss}}1
  {ü}{{\"u}}1
  {ä}{{\"a}}1
  {ö}{{\"o}}1
}
\begin{document}
\pointpoints{Punkt}{Punkte}
\bonuspointpoints{Bonuspunkt}{Bonuspunkte}
\renewcommand{\solutiontitle}{\noindent\textbf{Lösung:}%
\enspace}

\chqword{Frage}
\chpgword{Seite}
\chpword{Punkte}
\chbpword{Bonus Punkte}
\chsword{Erreicht}
\chtword{Gesamt}
\hpword{Punkte:} % Punktetabelle
\hsword{Ergebnis:}
\hqword{Aufgabe:}
\htword{Summe:}
\cellwidth{1.5em}
%\begin{center}
%\fbox{\fbox{\parbox{5.5in}{\centering
%Informatik-Klausur}}}
%\end{center}
%
%\vspace{5mm}
%
%\makebox[\textwidth]{Name:\enspace\hrulefill}
\pagestyle{headandfoot}
\runningheadrule

\newcommand\Vtextvisiblespace[1][.3em]{%
  \mbox{\kern.06em\vrule height.3ex}%
  \vbox{\hrule width#1}%
  \hbox{\vrule height.3ex}}

\newcommand{\klaubez}{Aufgaben zu Funktionen}
\firstpageheader{Informatik }{\klaubez} {\thepage /\numpages}
\runningheader{Informatik }{\klaubez} {\thepage /\numpages}
\newcommand{\pfad}{c:/Users/khthe/Dropbox/Informatik/KursV2/Python}
%-------------------------------------------------------------------
%\printanswers
%-------------------------------------------------------------------

\begin{questions}

\question[2] Schreibe eine Funktion func01, der ein String übergeben wird und
die den String ohne erstes und letztes Zeichen zurückgibt.
\begin{lstlisting}
>>> func01('Hallo')
'all'
\end{lstlisting}
\begin{solutionbox}{4cm}
\begin{lstlisting}
def func01(s):
    return s[1:-1]
\end{lstlisting}
\end{solutionbox}

\question[3] Schreibe eine Funktion func02, der ein nicht leerer String übergeben wird
und die den String wieder zurückgibt, aber erstes und letztes Zeichen sind vertauscht.
\begin{lstlisting}
>>> func02('Hallo')
'oallH'
\end{lstlisting}
\begin{solutionbox}{4cm}
\begin{lstlisting}
def func02(s):
    return s[-1]+s[1:-1]+s[0]
\end{lstlisting}
\end{solutionbox}

\question[3] Schreibe eine Funktion \texttt{func03}, der ein String übergeben wird und die
den String wieder zurückgibt ergänzt um den selben String in umgekehrter Reihenfolge.
\begin{lstlisting}
>>> func03('Hallo')
'HalloollaH'
\end{lstlisting}
\begin{solutionbox}{4cm}
\begin{lstlisting}
def func03(s):
    return s + s[::-1]
\end{lstlisting}
\end{solutionbox}

\question[3] Schreibe eine Funktion zaehl01, die einen nicht leeren String erhält und zurückgibt,
wieviel mal das erste Zeichen in dem String vorkommt. Die Stringmethode count darf nicht benutzt werden.
\begin{lstlisting}
>>> zaehl01('abbaaeacde')
4
\end{lstlisting}
\begin{solutionbox}{4cm}
\begin{lstlisting}
def zaehl01(s):
    zaehl = 0
    erster = s[0]
    for c in s:
        if c $==$ erster:
            zaehl+=1
    return zaehl

\end{lstlisting}
\end{solutionbox}

\question[3] Schreibe eine Funktion zaehl02, die einen nicht leeren String erhält und zurückgibt,
wieviele Ziffern (0-9) in dem String vorkommen.
\begin{lstlisting}
>>> zaehl02('ab12cc3d60bb')
5
\end{lstlisting}
\begin{solutionbox}{4cm}
\begin{lstlisting}
def zaehl02(s):
    zaehl = 0
    for c in s:
        if c in '0123456789':
            zaehl+=1
    return zaehl
\end{lstlisting}
\end{solutionbox}

\question[3] Schreibe eine Funktion zaehl03, die einen nicht leeren String und ein Zeichen
erhält und zurückgibt, wie oft das Zeichen an einem geraden Index des Strings vorkommt.
\begin{lstlisting}
>>> zaehl03('abaabbbba','a')
3
>>> zaehl03('abaabbbba','b')
2
\end{lstlisting}
\begin{solutionbox}{4cm}
\begin{lstlisting}
def zaehl03(s, c):
    zaehl = 0
    for i in range(len(s)):
        if i % 2 $==$ 0 and s[i] $==$ c:
            zaehl+=1
    return zaehl
\end{lstlisting}
\end{solutionbox}

\question[4] Schreibe eine Funktion zaehl01, die zurückgibt, wieviele Zahlen
es zwischen 100 und 999 (einschließlich) gibt, die durch 3 aber nicht durch 5
teilbar sind.
\begin{solutionbox}{4cm}
\begin{lstlisting}
def zaehl01():
    zaehl = 0
    for k in range(100,1000):
        if k % 3 == 0 and k % 5 != 0:
            zaehl+=1
    return zaehl
\end{lstlisting}
\end{solutionbox}

\question[4] Schreibe eine Funktion \texttt{zaehl02}, der zwei natürliche Zahlen
übergeben werden und die zurückgibt, wieviele Vielfache von 7 zwischen diesen
Zahlen (einschließlich) liegen.
\begin{lstlisting}
>>> zaehl02(4,17)
2
>>> zaehl02(0,21)
4
\end{lstlisting}
\begin{solutionbox}{4cm}
\begin{lstlisting}
def zaehl02(a, b):
    zaehl = 0
    for k in range(a, b+1):
        if k % 7 $==$ 0 :
            zaehl+=1
    return zaehl

\end{lstlisting}
\end{solutionbox}

\question[2] Schreibe eine Funktion pruef01, der ein String mit Länge >= 2
übergeben wird und die prüft, ob die ersten beiden Zeichen gleich sind.
\begin{lstlisting}
>>> pruef01('aab')
True
>>> pruef01('abb')
False
\end{lstlisting}
\begin{solutionbox}{4cm}
\begin{lstlisting}
def pruef01(s):
    return s[0] $==$ s[1]
\end{lstlisting}
\end{solutionbox}

\question[4] Schreibe eine Funktion pruef02, der ein String s , ein Zeichen c und eine
nicht negative ganze Zahl k übergeben wird, und die prüft, ob c in s k-mal vorkommt.
\begin{lstlisting}
>>> pruef02('aab','a',2)
True
>>> pruef02('aab','b',2)
False
\end{lstlisting}
\begin{solutionbox}{4cm}
\begin{lstlisting}
def pruef02(s, c, k):
    zaehl = 0
    for zeichen in s:
        if zeichen $==$ c:
            zaehl+=1
    return zaehl $==$ k
\end{lstlisting}
\end{solutionbox}

\question[4]

Das Programm erhält eine Ziffernfolge als Eingabe.
Es gibt die Summe aus, die sich aus den vorkommenden Ziffern 2 und 3 ergibt.

\begin{lstlisting}
Eingabe: 2343
8
\end{lstlisting}

\begin{solutionbox}{5cm}
\begin{lstlisting}
s = input('Eingabe: ')
summe = 0
for c in s:
    if c $==$ '2' or c $==$ '3':
       summe += int(c)
print(summe)
\end{lstlisting}
\end{solutionbox}

\question[4]

Das Programm erhält eine Ziffernfolge als Eingabe.
Es gibt die Summe aus, die sich aus allen Ziffern ergibt, die vor einer 3 stehen.

\begin{lstlisting}
Eingabe: 234543621
6
\end{lstlisting}

\begin{solutionbox}{5cm}
\begin{lstlisting}
s = input('Eingabe: ')

summe = 0
for i in range(1,len(s)):
    if s[i] $==$ '3':
       summe += int(s[i-1])
print(summe)
\end{lstlisting}
\end{solutionbox}

\question[4] Was erscheint auf der Konsole?  
\begin{lstlisting}
def a(x):
    return x + 1
def b(x):
    return x + 2
def c(x,y):
    return x + y
def d(x,y):
    return x > y
z = c(a(3),b(5))
print(d(z,10),z)
\end{lstlisting}
\begin{solutionbox}{1cm}
True 11
\end{solutionbox}

\question[4] Was erscheint auf der Konsole?  
\begin{lstlisting}
def a(x,y):
    return x + 2*y
def b(x):
    return x - 1
def c(x,y):
    return x - y
def d(x,y):
    return x - y > 0
z = c(a(3,4),b(2))
print(d(z,12),z)
\end{lstlisting}
\begin{solutionbox}{1cm}
 False 10
\end{solutionbox}

%\question[4] Was erscheint auf der Konsole?  
\begin{lstlisting}
def a(x,y):
    return 2 * x - y
def b(x):
    return x - 4
def c(x,y):
    return x * y
def d(x,y):
    return x + 4 * y > 0
z = c(a(1,7),b(6))
print(d(2 * z,8),z)
\end{lstlisting}
\begin{solutionbox}{1cm}
True -10
\end{solutionbox}

%\question[4] Was erscheint auf der Konsole?  
\begin{lstlisting}
def a(x):
    return x - 2
def b(x):
    return x + 4
def c(x,y):
    return x + 2*y
def d(x,y):
    return x > y
z = c(a(2),b(6))
print(d(z,8),z)
\end{lstlisting}
\begin{solutionbox}{1cm}
True 20
\end{solutionbox}

%\question[4] Was erscheint auf der Konsole?  
\begin{lstlisting}
def a(x,y):
    return x + 2 * y
def b(x):
    return x - 3
def c(x,y):
    return (x + 3) // y
def d(x,y):
    return x > b(y)
z = c(a(3,1),b(7))
print(d(z,5),z)
\end{lstlisting}
\begin{solutionbox}{1cm}
False 2
\end{solutionbox}





% -------------------------------------------------
\end{questions}
\begin{center}
%\pointtable[h][questions]
\end{center}

\end{document}
