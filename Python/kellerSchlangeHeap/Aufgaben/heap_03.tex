\question[4]
a. Gegeben sei eine Liste \texttt{a} mit Zahlen.
Erstelle daraus mit einer Comprehension eine Liste \texttt{b} von Tupeln,
die in einen Heap einfügt werden können. Als Vergleichswert dient der Abstand
der Zahl zur 42 (Funktion \texttt{abs}). D.h. bei Entnahme einer Zahl aus dem Heap
wird die Zahl mit dem kleinsten Abstand zur 42 entfernt. \\
b. Weise der Variablen \texttt{hp} einen leeren Heap zu. \\
c. Fülle \texttt{hp} in einer Schleife mittels typischer Operationen mit den Elementen von
\texttt{b}. \\
d. Gehe in der typischen Schleife durch den Heap, entferne jeweils ein Tupel aus dem Heap
und gib die in dem Tupel enthaltene Zahl von a aus.

\begin{solutionbox}{5cm}
\begin{lstlisting}
from heapq import heappop, heappush
b = [(abs(x-42),x) for x in a]
hp = []
for s in b:
    heappush(hp,s)

while hp:
    print(heappop(hp)[1])
\end{lstlisting}
\end{solutionbox}
