\documentclass[addpoints]{exam}
\usepackage[utf8x]{inputenc}
\usepackage[ngerman]{babel}
\usepackage{listings}
\usepackage{babel}
\usepackage[top=1.5cm,bottom=0.5cm,headsep=0.5cm,headheight=3cm,%
left=1.5cm,right=1.5cm]{geometry}

\usepackage[T1]{fontenc}
\usepackage{booktabs} % schöne Tabellen
\usepackage{graphicx}
\usepackage{csquotes} % Anführungszeichen
\usepackage{paralist} % kompakte Aufzählungen
\usepackage{amsmath,textcomp,tikz} %diverses
\usepackage{eso-pic} % Bilder im Hintergrund
\usepackage{mdframed} % Boxen
\usepackage{multirow}


\newmdenv[linecolor=black,backgroundcolor=gray!15,
frametitle={Punktverteilung},leftmargin=1cm,
rightmargin=1cm]{infobox}

\lstset{language=Python, tabsize=4, basicstyle=\footnotesize, showstringspaces=false, mathescape=true,commentstyle=\fontfamily{fve}  }
\lstset{literate=%
  {Ö}{{\"O}}1
  {Ä}{{\"A}}1
  {Ü}{{\"U}}1
  {ß}{{\ss}}1
  {ü}{{\"u}}1
  {ä}{{\"a}}1
  {ö}{{\"o}}1
}
\begin{document}
\pointpoints{Punkt}{Punkte}
\bonuspointpoints{Bonuspunkt}{Bonuspunkte}
\renewcommand{\solutiontitle}{\noindent\textbf{Lösung:}%
\enspace}

\chqword{Frage}
\chpgword{Seite}
\chpword{Punkte}
\chbpword{Bonus Punkte}
\chsword{Erreicht}
\chtword{Gesamt}
\hpword{Punkte:} % Punktetabelle
\hsword{Ergebnis:}
\hqword{Aufgabe:}
\htword{Summe:}
\cellwidth{1.5em}
%\begin{center}
%\fbox{\fbox{\parbox{5.5in}{\centering
%Informatik-Klausur}}}
%\end{center}
%
%\vspace{5mm}
%
%\makebox[\textwidth]{Name:\enspace\hrulefill}
\pagestyle{headandfoot}
\runningheadrule

\newcommand\Vtextvisiblespace[1][.3em]{%
  \mbox{\kern.06em\vrule height.3ex}%
  \vbox{\hrule width#1}%
  \hbox{\vrule height.3ex}}

\newcommand{\klaubez}{Aufgaben zu Stack, Queue, Heap}
\firstpageheader{Informatik }{\klaubez} {\thepage /\numpages}
\runningheader{Informatik }{\klaubez} {\thepage /\numpages}
\newcommand{\pfad}{c:/Users/khthe/Dropbox/Informatik/KursV2/Python}
%-------------------------------------------------------------------
\printanswers
%-------------------------------------------------------------------

\begin{questions}

\question[4]
a. Weise der Variablen \texttt{st} einen leeren Stack zu. \\
b. Fülle den Stack (mittels typischer Operationen)
nacheinander mit den Zahlen 5, 42, 17. \\
c. Gehe in der typischen Schleife durch den Stack und gib den Inhalt aus.
\begin{solutionbox}{5cm}
\begin{lstlisting}
st = []
st.append(5)
st.append(42)
st.append(17)
while st:
    print(st.pop())
\end{lstlisting}
\end{solutionbox}

\question[3] a. Weise der Variablen \texttt{qu} eine leere Queue zu. \\
b. Fülle die Queue (mittels typischer Operationen)
nacheinander mit den Zahlen 5, 42, 17. \\
c. Gehe in der typischen Schleife durch die Queue und gib den Inhalt aus.
\begin{solutionbox}{5cm}
\begin{lstlisting}
from collections import deque
qu = deque()
qu.append(5)
qu.append(42)
qu.append(17)
while qu:
    print(qu.popleft())
\end{lstlisting}
\end{solutionbox}

\question[5]
a. Gegeben sei das dictionary \texttt{m = \{'a':5, 'b':42, 'c':17\}}
Erstelle daraus mit einer Comprehension eine Liste \texttt{w} von Tupeln,
die in einen Heap einfügt werden können (mit der Zahl als Vergleichswert).  \\
b. Weise der Variablen \texttt{hp} einen leeren Heap zu. \\
c. Fülle \texttt{hp} (mittels typischer Operationen) mit den Elementen von
\texttt{w}. \\
d. Gehe in der typischen Schleife durch den Heap und gib den Inhalt aus.
\begin{solutionbox}{5cm}
\begin{lstlisting}
from heapq import heapify, heappop, heappush
m = {'a':5, 'b':42, 'c':17}
w = [(m[x],x) for x in m]
hp = []
for t in w:
    heappush(hp,t)

while hp:
    print(heappop(hp))
\end{lstlisting}
\end{solutionbox}


% -------------------------------------------------
\end{questions}
\begin{center}
%\pointtable[h][questions]
\end{center}

\end{document}
