\documentclass[addpoints]{exam}
\usepackage[utf8x]{inputenc}
\usepackage[ngerman]{babel}
\usepackage{listings} 
\usepackage{babel}
\usepackage[top=1.5cm,bottom=0.5cm,headsep=0.5cm,headheight=3cm,%   
left=1.5cm,right=1.5cm]{geometry}                      
                         
\usepackage[T1]{fontenc}                               
\usepackage{booktabs} % schöne Tabellen                
\usepackage{graphicx}                                  
\usepackage{csquotes} % Anführungszeichen              
\usepackage{paralist} % kompakte Aufzählungen          
\usepackage{amsmath,textcomp,tikz} %diverses          
\usepackage{eso-pic} % Bilder im Hintergrund          
\usepackage{mdframed} % Boxen         
\usepackage{multirow}    

    
\newmdenv[linecolor=black,backgroundcolor=gray!15,    
frametitle={Punktverteilung},leftmargin=1cm,
rightmargin=1cm]{infobox}

\lstset{language=Python, tabsize=4, basicstyle=\footnotesize, showstringspaces=false, mathescape=true}  
\lstset{literate=%
  {Ö}{{\"O}}1
  {Ä}{{\"A}}1
  {Ü}{{\"U}}1
  {ß}{{\ss}}1
  {ü}{{\"u}}1
  {ä}{{\"a}}1
  {ö}{{\"o}}1
}
\begin{document}
\pointpoints{Punkt}{Punkte}
\bonuspointpoints{Bonuspunkt}{Bonuspunkte}                     
\renewcommand{\solutiontitle}{\noindent\textbf{Lösung:}%       
\enspace}                                                      
                                                               
\chqword{Frage}                                                
\chpgword{Seite}                                               
\chpword{Punkte}                                               
\chbpword{Bonus Punkte}                                        
\chsword{Erreicht}                                            
\chtword{Gesamt}                                              
\hpword{Punkte:} % Punktetabelle                              
\hsword{Ergebnis:}                                            
\hqword{Aufgabe:}                                             
\htword{Summe:}      
\cellwidth{1.5em}                                         
%\begin{center}
%\fbox{\fbox{\parbox{5.5in}{\centering
%Informatik-Klausur}}}
%\end{center}
%
%\vspace{5mm}
%
%\makebox[\textwidth]{Name:\enspace\hrulefill}
\pagestyle{headandfoot}
\runningheadrule

\newcommand\Vtextvisiblespace[1][.3em]{%
  \mbox{\kern.06em\vrule height.3ex}%
  \vbox{\hrule width#1}%
  \hbox{\vrule height.3ex}}

\newcommand{\klaubez}{Aufgaben zu Strings, Ein-/Ausgabe, Verzweigungen, Schleifen}
\firstpageheader{Informatik }{\klaubez} {\thepage /\numpages}
\runningheader{Informatik }{\klaubez} {\thepage /\numpages}
\newcommand{\pfad}{c:/Users/khthe/Dropbox/Informatik/KursV2/Python}
%-------------------------------------------------------------------
%\printanswers
%-------------------------------------------------------------------

\begin{questions}
\question[7]
Zu was werten sich die Ausdrücke aus? Wenn die Auswertung zu einem Fehler führt, notiere das
Wort 'error'.

\begin{lstlisting}
a. "a" + "bc"    b. 3 * "bc"   c. "3" * "bc"   d. "abcd"[2]   e. "abcd"[1:2]
f. "abcd"[:2]    g. "abcd"[2:]
\end{lstlisting}
\begin{solutionbox}{2cm}
\begin{lstlisting}
a. "abc"   b. "bcbcbc"  c. error   d. "c   e. "c"   f. "ab"   g. "cd"
\end{lstlisting}
\end{solutionbox}

\question[7]
Nimm an, wir haben folgende Zuweisungen gemacht:
\begin{lstlisting}
s1 = 'abcde'
s2 = '4567'
\end{lstlisting}
Zu was werten sich die folgenden Ausdrücke aus? Wenn die Auswertung zu einem Fehler führt, notiere das
Wort 'error'.

\begin{lstlisting}
a. s1[-2]   b. len(s2+s1)  c. s1 + len(s2)  d. s1[-len(s2)]
e. (s1+s2)[2:]   f. s1 + s2[:2]   g. s1[:1:-1]

\end{lstlisting}
\begin{solutionbox}{2cm}
\begin{lstlisting}
a. 'd'   b. 9   c. error  d. 'b'  e. 'cde4567'  f. 'abcde45'  g. 'edc'
\end{lstlisting}
\end{solutionbox}

\question[3] 
Was erscheint auf der Konsole, wenn das Programm nacheinander mit folgenden
Eingaben gestartet wird?
a. 12  ~~ b. 5  ~~ c. 2
\begin{lstlisting}
a = int(input())
if a > 20:
    print('A')
elif a >= 5:
    print('B')
elif a > 10:
    print('C')
else:
    print('D')
\end{lstlisting}
\begin{solutionbox}{1cm}
a. B ~~ b. B ~~ c. D
\end{solutionbox}

\question[3] Was erscheint auf der Konsole, wenn das Programm nacheinander mit folgenden Eingaben gestartet wird?
a. 15   ~~  b. 35 ~~ c. 50
\begin{lstlisting}
x = int(input())
if x > 10:
    print('A')
elif x > 20:
    print('B')
if x > 30:
    print('C')
if x < 40:
    print('D')
else:
    print('E')
\end{lstlisting}

\begin{solutionbox}{1cm}
a. A, D ~~ b. A, C, D ~~ c. A, C, E
\end{solutionbox}

\question[3] Welche Werte durchläuft i ?
\begin{lstlisting}
a. for i in range(4):
b. for i in range(4,6,2):
c. for i in range(3,7):
d. for i in range(8,4):
e. for i in range(8,4,-2):
f. for i in range(2,-2,-1):
\end{lstlisting}
\begin{solutionbox}{3cm}
a. 0, 1 , 2, 3 ~b. 4 ~ c. 3, 6 ~ d. kein Wert  ~ e. 8, 6 ~ f. 2, 1, 0, -1
\end{solutionbox}

\question[3] Welche Werte durchläuft i ? 
\begin{lstlisting}
a. for i in range(5):
b. for i in range(-2, 5):
c. for i in range(6, 4, -2):
d. for i in range(5, 3):
e. for i in range(5, 0, -1):
f. for i in range(2, 3):
\end{lstlisting}
\begin{solutionbox}{3cm}
a. 0, 1, 2, 3, 4 ~ b. -2, -1, 0, 1, 2, 3, 4 ~ c. 6   \\
d. kein Wert, ~ e. 5, 4, 3, 2, 1 ~ f. 2
\end{solutionbox}

\question[4] Was erscheint auf der Konsole?
\begin{lstlisting}
for i in range(4,10):
    if i % 2 $==$ 0:
        print('a',end='')
    if i % 4 $==$ 0:
        print('b',end='')
    else:
        print('c',end='')
\end{lstlisting}
\begin{solutionbox}{1cm}
\texttt{abcaccabc}
\end{solutionbox}

\question[4] Was erscheint auf der Konsole?
\begin{lstlisting}
zaehl = 0
zahl = 10
while zaehl < 10:
    print(zahl,end=' ')
    if zaehl % 2 $==$ 0:
        zahl+=5
    else:
        zahl+=3
    zaehl+=1
\end{lstlisting}
\begin{solutionbox}{1cm}
\texttt{10 15 18 23 26 31 34 39 42 47 }
\end{solutionbox}

\question[4] Was erscheint auf der Konsole? \\
\begin{minipage}[c]{6cm}
\begin{lstlisting}
a.
for i in range(10):
    print(i,end=' ')
    if i > 4: break
print("Ende1")
\end{lstlisting}
\end{minipage}
\begin{minipage}[c]{6cm}
\begin{lstlisting}
b.
j = 0
while j < 10:
    j = j + 2
    if j % 3 $==$ 0:
        j+=1
        continue
    print(j,end=' ')
print("Ende2")
\end{lstlisting}
\end{minipage}
\begin{solutionbox}{2cm}
\begin{lstlisting}
a. 0 1 2 3 4 5 Ende1
b. 2 4 Ende2
\end{lstlisting}
\end{solutionbox}

%break and coninue A2 --------------------------
\question[4] Was erscheint auf der Konsole?

\begin{minipage}[c]{6cm}
\begin{lstlisting}
a.
for i in range(10):
    if i $==$ 5:
        break
    print(i,end=' ')
print("Ende1")
\end{lstlisting}
\end{minipage}
\begin{minipage}[c]{6cm}
\begin{lstlisting}
b.
j = 0
while j < 10:
    j+=1
    if j % 2 $==$ 0:
        continue
    print(j,end=' ')
print("Ende2")
\end{lstlisting}
\end{minipage}
\begin{solutionbox}{2cm}
\begin{lstlisting}
a. 0 1 2 3 4 Ende1
b. 1 3 5 7 9 Ende2
\end{lstlisting}
\end{solutionbox}

\question[5]
Der Anwender soll eine ganze Zahl zwischen 1 und 10 eingeben (1,2,...,10).
Bei der Eingabe von 1,2,7 oder 9 erscheint die Meldung
\texttt{Ausgabe A}. Bei der Eingabe von 3 oder 5 erscheint die Meldung
\texttt{Ausgabe B}. Bei den restlichen zulässigen Zahlen erscheint die Meldung
\texttt{Ausgabe C}. Wird eine Zahl eingegeben, die nicht zwischen 1 und 10 liegt, erscheint
die Meldung \texttt{ungültige Zahl}. Der Fall, dass keine Zahl eingegeben wird, muss nicht behandelt werden.

\begin{solutionbox}{8cm}
\begin{lstlisting}
x = int(input('Bitte Zahl zwischen 1 und 10 eingeben: '))
if x == 1 or x == 2 or x == 7 or x == 9:
    print('Ausgabe A')
elif x == 3 or x == 5:
    print('Ausgang B')
elif x == 4 or x == 6 or x == 8 or x == 10:
    print('Ausgang C')
else:
    print('ungültige Zahl')
\end{lstlisting}
\end{solutionbox}

\question[6]
Abhängig von der Eingabe der Wochentagszahl soll die Anzahl der jeweiligen Schulstunden ausgegeben werden.
Als Wochentagszahl legen wir fest: \\
1 - Montag, 2 - Dienstag, ... , 7 - Sonntag \\
Die Schulstunden sind wie folgt verteilt:
Montag: 8 Stunden - Dienstag, Mittwoch, Freitag: 6 Stunden - Donnerstag: 5 Stunden.
Bei Samstag und Sonntag soll die Meldung \texttt{Wochenende} erscheinen, sonst die Meldung \texttt{ungültiger Wochentag}
\begin{solutionbox}{8cm}
\begin{lstlisting}
x = int(input('Bitte Wochentagszahl eingeben: '))
if x $==$ 1:
    print(8)
elif x $==$ 2 or x $==$ 3 or x $==$ 5:
    print(6)
elif x $==$ 4:
    print(5)
elif x $==$ 6 or x $==$ 7:
    print('Wocheende')
else:
    print('ungültiger Wochentag')
\end{lstlisting}
\end{solutionbox}

\question[4]
Das Programm erhält zwei Zahlen als Eingabe und untersucht den Zahlbereich zwischen den
beiden Zahlen (Grenzen eingeschlossen):
Es zählt die Vielfachen von 5, die keine Vielfachen von 4 sind.

\begin{lstlisting}
Eingabe1: 1
Eingabe2: 22
3
\end{lstlisting}

\begin{solutionbox}{5cm}
\begin{lstlisting}
n = int(input('Eingabe1: '))
m = int(input('Eingabe2: '))

zaehl = 0
for k in range(n,m+1):
    if k % 5 $==$ 0 and k % 4 != 0:
         zaehl+=1
print(zaehl)
\end{lstlisting}
\end{solutionbox}

\question[4]
Das Programm erhält zwei Zahlen als Eingabe und untersucht den Zahlbereich zwischen den
beiden Zahlen (Grenzen eingeschlossen).
Es gibt alle Zahlen aus, die Vielfache von 7 sind und die Ziffernfolge
21 enthalten. Als letzte Ausgabe wird die Anzahl der gefundenen Zahlen ausgegeben.

\begin{lstlisting}
Eingabe1: 1
Eingabe2: 1000
21
210
217
721
4
\end{lstlisting}

\begin{solutionbox}{5cm}
\begin{lstlisting}
n = int(input('Eingabe1: '))
m = int(input('Eingabe2: '))

zaehl = 0
for k in range(n,m+1):
    if k % 7 $==$ 0 and '21' in str(k):
         print(k)
         zaehl+=1
print(zaehl)
\end{lstlisting}
\end{solutionbox}

\question[4]

Das Programm erhält eine Ziffernfolge als Eingabe.
Es gibt die Summe aus, die sich aus den vorkommenden Ziffern 2 und 3 ergibt.

\begin{lstlisting}
Eingabe: 2343
8
\end{lstlisting}

\begin{solutionbox}{5cm}
\begin{lstlisting}
s = input('Eingabe: ')
summe = 0
for c in s:
    if c $==$ '2' or c $==$ '3':
       summe += int(c)
print(summe)
\end{lstlisting}
\end{solutionbox}

\question[4]

Das Programm erhält eine Ziffernfolge als Eingabe.
Es gibt die Summe aus, die sich aus allen Ziffern ergibt, die vor einer 3 stehen.

\begin{lstlisting}
Eingabe: 234543621
6
\end{lstlisting}

\begin{solutionbox}{5cm}
\begin{lstlisting}
s = input('Eingabe: ')

summe = 0
for i in range(1,len(s)):
    if s[i] $==$ '3':
       summe += int(s[i-1])
print(summe)
\end{lstlisting}
\end{solutionbox}




% -------------------------------------------------
\end{questions}
\begin{center}
%\pointtable[h][questions]
\end{center}

\end{document}