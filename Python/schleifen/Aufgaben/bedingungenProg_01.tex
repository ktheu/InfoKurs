\question[5]
Der Anwender soll eine ganze Zahl zwischen 1 und 10 eingeben (1,2,...,10).
Bei der Eingabe von 1,2,7 oder 9 erscheint die Meldung
\texttt{Ausgabe A}. Bei der Eingabe von 3 oder 5 erscheint die Meldung
\texttt{Ausgabe B}. Bei den restlichen zulässigen Zahlen erscheint die Meldung
\texttt{Ausgabe C}. Wird eine Zahl eingegeben, die nicht zwischen 1 und 10 liegt, erscheint
die Meldung \texttt{ungültige Zahl}. Der Fall, dass keine Zahl eingegeben wird, muss nicht behandelt werden.

\begin{solutionbox}{8cm}
\begin{lstlisting}
x = int(input('Bitte Zahl zwischen 1 und 10 eingeben: '))
if x == 1 or x == 2 or x == 7 or x == 9:
    print('Ausgabe A')
elif x == 3 or x == 5:
    print('Ausgang B')
elif x == 4 or x == 6 or x == 8 or x == 10:
    print('Ausgang C')
else:
    print('ungültige Zahl')
\end{lstlisting}
\end{solutionbox}
