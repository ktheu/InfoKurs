\documentclass[addpoints]{exam}
\usepackage[utf8x]{inputenc}
\usepackage[ngerman]{babel}
\usepackage{listings}
\usepackage{babel}
\usepackage[top=1.5cm,bottom=0.5cm,headsep=0.5cm,headheight=3cm,%
left=1.5cm,right=1.5cm]{geometry}

\usepackage[T1]{fontenc}
\usepackage{booktabs} % schöne Tabellen
\usepackage{graphicx}
\usepackage{csquotes} % Anführungszeichen
\usepackage{paralist} % kompakte Aufzählungen
\usepackage{amsmath,textcomp,tikz} %diverses
\usepackage{eso-pic} % Bilder im Hintergrund
\usepackage{mdframed} % Boxen
\usepackage{multirow}


\newmdenv[linecolor=black,backgroundcolor=gray!15,
frametitle={Punktverteilung},leftmargin=1cm,
rightmargin=1cm]{infobox}

\lstset{language=Python, tabsize=4, basicstyle=\footnotesize, showstringspaces=false, mathescape=true}
\lstset{literate=%
  {Ö}{{\"O}}1
  {Ä}{{\"A}}1
  {Ü}{{\"U}}1
  {ß}{{\ss}}1
  {ü}{{\"u}}1
  {ä}{{\"a}}1
  {ö}{{\"o}}1
}
\begin{document}
\pointpoints{Punkt}{Punkte}
\bonuspointpoints{Bonuspunkt}{Bonuspunkte}
\renewcommand{\solutiontitle}{\noindent\textbf{Lösung:}%
\enspace}

\chqword{Frage}
\chpgword{Seite}
\chpword{Punkte}
\chbpword{Bonus Punkte}
\chsword{Erreicht}
\chtword{Gesamt}
\hpword{Punkte:} % Punktetabelle
\hsword{Ergebnis:}
\hqword{Aufgabe:}
\htword{Summe:}
\cellwidth{1.5em}
%\begin{center}
%\fbox{\fbox{\parbox{5.5in}{\centering
%Informatik-Klausur}}}
%\end{center}
%
%\vspace{5mm}
%
%\makebox[\textwidth]{Name:\enspace\hrulefill}
\pagestyle{headandfoot}
\runningheadrule

\newcommand\Vtextvisiblespace[1][.3em]{%
  \mbox{\kern.06em\vrule height.3ex}%
  \vbox{\hrule width#1}%
  \hbox{\vrule height.3ex}}

\newcommand{\klaubez}{Aufgaben zu Graphen1}
\firstpageheader{Informatik }{\klaubez} {\thepage /\numpages}
\runningheader{Informatik }{\klaubez} {\thepage /\numpages}
\newcommand{\pfad}{c:/MyData/07_Kurse/Kursaufgaben}
%-------------------------------------------------------------------
%\printanswers
%-------------------------------------------------------------------

\begin{questions}
\question[3]
Gib für den abgebildeten Graphen die Implementierung in Python als
ein dictionary von sets an.

\includegraphics[height=3cm]{\pfad/Graphen/Aufgaben/ungewichtet_01/ungewichtet_01.png}
\begin{solutionbox}{5cm}
\begin{lstlisting}
G = {
'a': set('bcdef'),
'b': set('ce'),
'c': set('d'),
'd': set('e'),
'e': set('f'),
'f': set('cgh'),
'g': set('fh'),
'h': set('fg')
}
\end{lstlisting}
\end{solutionbox}

\question[3]
Gib für den abgebildeten Graphen die Implementierung in Python als
ein dictionary von sets an.

\includegraphics[height=3cm]{\pfad/Graphen/Aufgaben/ungewichtet_03/ungewichtet_03.png}
\begin{solutionbox}{5cm}
\begin{lstlisting}
G = {
'a': set('bcd'),
'b': set('ca'),
'c': set('ade'),
'd': set('ace'),
'e': set('cd')
}
\end{lstlisting}
\end{solutionbox}


\newpage
\input{\pfad/Graphen/gewichtet_01/gewichtet_01}
\input{\pfad/Graphen/reihenfolge_02/reihenfolge_02}
\newpage
\question[3]
Zeichne den shortest-path Baum der Breitensuche. Wir beginnen bei a und nehmen an,
dass Knoten auf der gleichen Ebene in alphabetischer Reihenfolge besucht werden.

\includegraphics[height=4cm]{\pfad/Graphen/Aufgaben/breitensuche_02/breitensuche_02.png}
\begin{solutionbox}{5cm}
\includegraphics[height=4cm]{\pfad/Graphen/Aufgaben/breitensuche_02/breitensuche_baum_02.png}
\end{solutionbox}

\question[6]
Ermittle mit dem Algorithmus von Dijkstra die kürzesten Wege von Knoten a zu allen anderen
Knoten.  Notiere die Reihenfolge der endgültig markierten Knoten.
Notiere für jeden Knoten die Reihenfolge der Werte, mit denen er markiert wird.

\includegraphics[height=4cm]{\pfad/Graphen/Aufgaben/dijkstra_02/dijkstra_02.png}
\begin{solutionbox}{5cm}
\begin{lstlisting}
Reihenfolge der endgültigen Markierungen: a b d e f c
a : 0
b : inf 3
c : inf 10 9
d : inf 6
e : inf 8 7
f : inf 8 7
\end{lstlisting}
\end{solutionbox}

\newpage
\input{\pfad/Graphen/grid01/grid01}
\newpage
\input{\pfad/Graphen/dijkstra_heap01/dijkstra_heap01}

% -------------------------------------------------
\end{questions}
\begin{center}
%\pointtable[h][questions]
\end{center}

\end{document}
